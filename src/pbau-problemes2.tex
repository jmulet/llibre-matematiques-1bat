\documentclass[11pt, a4paper, pdf]{book}
 
%\include{estil-llibre}
\usepackage{iesbbook}
  
\begin{document} 
 
%\chapter*{Selectivitat  Juny 2017}


%%%%%%%%%%%%%%%%%%%%%%%%%%%%%%%%%%%%%%%%%%%%%%%%%%%%%%%%%%%%%%%%%%%%%%%%%%%%%%%%%%%%%%%%%%%%%%%%%%%%%%%%%%%%%%%%%%%%%%%%%%%%%%%

\section*{JUNY 2017 - Opció A} 

\begin{mylist}
	
\item \begin{enumerate}
	\item Discutiu per a quins valors de $m$ el sistema següent és compatible.
	\[  \left. \begin{array}{ll} mx+3z=m \\ x+2y-z=1 \\ 2x+y-z=2  \end{array} \right\} \]
	
	\item Resoleu-lo en el cas o casos que sigui compatible indeterminat.

\end{enumerate}

\item El nombre de litres per metre quadrat que va ploure en un determinat lloc ve donat per la següent funció:
\[  Q(t) =-\frac{t^3}{8}+ \frac{3t^2}{2}-\frac{9t}{2} +10 \]
	on $t$ ve donat en dies i va des de el dia $t=1$ (dilluns) fins al dia $t=8$ (dilluns de l'altre setmana).
	\begin{enumerate}
		\item Trobeu el dia de la setmana que va ploure més  i el que va ploure menys. Quants de litres per metre quadrat
		va ploure durant aquests dos dies? 
		
		\item Feu un petit dibuix  de la funció anterior durant els 8 dies. 
	\end{enumerate} 

 
\item Llançam dos daus de 6 cares no trucats i consideram els esdeveniments següents:

\quad 
\quad $S_7$: ``La suma dels resultats dels dos daus és 7''


\quad 
\quad $P$: ``El producte dels resultats dels dos daus és imparell''

\begin{enumerate}
	\item Calculau les probabilitats que passin els esdeveniments anteriors.
	\item Són independents $S_7$ i $P$? Raonau la resposta.
\end{enumerate}


\end{mylist}




\section*{JUNY 2017 - Opció B} 
\setcounter{myenumi}{0}
\begin{mylist}

\item Tenim tres aixetes per omplir un dipòsit d'aigua on suposem que el caudal cau per cada aixeta és constant. Si fem servir l'aixeta 1, tardam 10 hores en omplir el dipòsit, si fem servir les aixetes 1 i 2, tardam 4 hores i si les feim servir totes tres, tardam una hora.
Suposant que la suma dels caudals de les tres aixetes és 10 litres per minut, trobeu el caudal de l'aigua de cada aixeta i el volum del dipòsit.


\vspace{-2cm}
\item \begin{minipage}[t]{0.7\textwidth}
	Hem de dissenyar una finestra com la que surt a la figura adjunta, o sigui, el polígon $ACEDB$ de 30 m de perímetre. Es tracta d'un rectangle amb un triangle equilàter damunt. Trobeu les dimensions del rectangle perquè l'àrea de la finestra sigui màxima.
	
\end{minipage}
\begin{minipage}{0.3\textwidth}
	\centering
	\vspace{1.5cm}
	\includegraphics[width=0.99\textwidth]{pbau17b02}
\end{minipage}
 
\end{mylist}

\end{document}