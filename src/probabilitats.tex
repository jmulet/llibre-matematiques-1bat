\begin{enumerate}
\def\labelenumi{\arabic{enumi}.}
\tightlist
\item
  Escriu el conjunt de possibles resultats de l'experiment aleatori:
  ``\emph{Escriure en cinc targetes cadascuna de les vocals i treure una
  a l'atzar}''.
\item
  Escriu tres successos aleatoris de l'experiment aleatori treure una
  carta d'una baralla espanyola. 
\item
  Sigui A el succés tirar un dau i treure un nombre major que 4. Escriu
  el succés contrari de A .
\item
  Calcula la probabilitat que en treure una carta de la baralla sigui
  una espasa.
\item
  Quina és la probabilitat de \emph{no} treure un 5 en tirar un dau? I
  de \emph{no} treure un múltiple de 3? I de \emph{no} treure un nombre
  menor que 2? 
\item
  En tirar una moneda dues vegades, quin és la probabilitat de no treure
  cap cara? I de treure almenys una cara? Observa que treure almenys una
  cara és el succés contrari de no treure cap cara.
\item
  En l'experiment ``treure tres cartes seguides'', quina és la
  probabilitat de \emph{treure tres asos}? Primer amb reemplaçament, i
  després sense reemplaçament.
\item
  En tirar dues vegades un dau calcula la probabilitat que surti un sis
  doble. 
\item
  En tirar dues vegades un dau calcula la probabilitat de treure almenys
  un 6. \emph{Ajuda}: Potser et sigui més fàcil calcular la probabilitat
  de \emph{no treure cap} 6, i utilitzar el succés contrari.
\item
  Llancem dos daus que no estiguin trucats i anotem els nombres de la
  seva cara superior. Considerem el succés \emph{A que} la suma de les
  dues cares sigui 8, i el succés \emph{B} que aquests nombres
  difereixin en dues unitats. a) Comprova que \emph{P}(\emph{A}) = 5/36
  (\emph{casos favorables}: 2 + 6; 3 + 5; 4 + 4; 5 + 3; 6 + 2) i que
  \emph{P}(\emph{B}) = 8/36 (\emph{casos favorables}: (1, 3), (2, 4),
  \ldots{}). 
\item
  Dibuixa en el teu quadern un diagrama en arbre per a tres incendis, i
  calcula la probabilitat que almenys un hagi estat intencionat essent
  la probabilitat d'un incendi \emph{P}(\emph{I}) = 0'6.
\item
  En una aeronau s'han instal·lat tres dispositius de seguretat:
  \emph{A} \emph{, B} i \emph{C}. Si falla \emph{A} es posa \emph{B} en
  funcionament, i si també falla \emph{B} comença a funcionar \emph{C}.
  Les probabilitats que funcioni correctament cada dispositiu són:
  \emph{P}(\emph{A}) = 0'96; \emph{P}(\emph{B}) = 0'98 i P (\emph{C}) =
  0'99. a) Calcula la probabilitat que fallin els tres dispositius. b)
  Calcula la probabilitat que tot vagi bé.
\item
  Una fàbrica de joguines rebutja normalment el 0'3 \% de la seva
  producció per fallades degudes a l'atzar. Calcula la probabilitat que:
  a) En agafar dues joguines a l'atzar calgui rebutjar ambdues. b) En
  agafar dues joguines a l'atzar calgui rebutjar només una. c) En agafar
  dues joguines a l'atzar no calgui rebutjar cap. 
\item
  Es tenen 3 caixes, A, B i C. La caixa A té 10 boles de les quals 4 són
  negres. La caixa B té 6 boles amb una bola negra. La caixa C té 8
  boles amb 3 negres. S'agafa una caixa a l'atzar i d'aquesta caixa es
  treu una bola, també a l'atzar. Comprova que la probabilitat que la
  bola sigui negra és 113/360.
\item
  Tenim una moneda trucada la probabilitat de la qual d'obtenir cara és
  3/5 i la de creu és 2/5. Si surt cara s'escull a l'atzar un nombre de
  l'1 al 8, i si surt creu, s'escull un nombre de l'1 al 6. Calcula la
  probabilitat que el nombre escollit sigui imparell. I PR
\item
  Antoni, Joan i Jordi tenen una prova de natació. Antoni i Joan tenen
  la mateixa probabilitat de guanyar, i doble a la probabilitat de
  Jordi. Calcula la probabilitat que guanyi Joan o Jordi.
\item
  Llancem dos daus i anotem els valors de les cares superiors. Calcula
  les probabilitats que la suma sigui 1, sigui 2, sigui 3, \ldots{}.
  sigui 12.
\item
  Llancem una moneda 50 vegades, què és més probable, obtenir 50 cares
  seguides o obtenir en les primeres 25 tirades cara i en les 25
  següents creu? Raona la resposta.
\item
  Una moneda està trucada. La probabilitat d'obtenir cara és doble que
  la d'obtenir creu. Calcula les probabilitats dels successos obtenir
  cara i d'obtenir creu en tirar la moneda.
\item
  Tenim un dau trucat de manera que els nombres imparells tenen una
  probabilitat doble a la dels nombres parells. Calcula les
  probabilitats de: A) Surti un nombre imparell. B) Surti un nombre
  primer. C) Surti un nombre primer imparell. D) Surti un nombre que
  sigui primer o sigui imparell.
\item
  En un grup de 12 amigues hi ha 3 rosses. Es trien dues noies a
  l'atzar. Calcula la probabilitat que: A) Ambdues siguin rosses. B)
  Almenys una sigui rossa. C) Cap sigui rossa. D) Una sigui rossa i
  l'altra no.
\item
  Llancem dos daus i anotem els valors de les cares superiors. Calcula
  les probabilitats que: A) Els nombres obtinguts siguin iguals. B) Els
  nombres obtinguts difereixin en 3 unitats. C) Els nombres obtinguts
  siguin parells.
\item
  Llancem una moneda fins que surti cara. Calcula la probabilitat que:
  A) Surti cara abans del quart llançament. B) Surti cara després del
  vuitè llançament.
\item
  Un lot de 20 articles té 2 defectuosos. Es treuen 4 a l'atzar, quin és
  la probabilitat que cap sigui defectuós?
\end{enumerate}

ACTIVITATS PROPOSADES

DISTRIBUCIONS DE PROBABILITAT DISCRETES. LA BINOMIAL

\begin{enumerate}
\def\labelenumi{\arabic{enumi}.}
\item
  En una distribució binomial \emph{B}(10, 0'3) calcula
  \emph{P}(\emph{x} = 0), P (\emph{x}  0), P (\emph{x} = 10) i P
  (\emph{x} = 7). Determina també la mitjana i la desviació típica.
\item
  Llancem 5 monedes, calcula les probabilitats d'obtenir:

  a) 0 cares, b) 1 cara, c) 2 cares, d) 3 cares
\item
  Es llança un dau tres vegades i es compta el nombres de tresos que
  apareixen. Es considera la variable aleatòria ``nombre de tresos
  obtinguts''. Representa la distribució de probabilitat. Calcula la
  mitjana i la desviació típica.
\item
  La població activa d'un cert país es pot dividir en els quals tenen
  estudis superiors i els que no els tenen, sent el primer d'un 20\%.
  Triem 10 persones de la població activa a l'atzar. Escriu l'expressió
  de totes les possibilitats i les seves probabilitats. Calcula la
  probabilitat que hi hagi 9 o 10 que tinguin estudis superiors.
\item
  S'estima que el percentatge de llars que utilitza una determinada
  marca de tomàquet fregit és del 12\%. En una mostra de 20 llars, quina
  probabilitat hi ha de trobar entre 6 i 15 que ho utilitzin? (No ho
  calculis, només planteja com ho calcularies).
\item
  Una escola té 500 alumnes 20 dels quals són esquerrans. N'elegim tres
  a l'atzar. Quina és la probabilitat que almenys un sigui esquerrà?
  Suposeu que en cada elecció d'un alumne, la probabilitat que sigui
  esquerrà és la mateixa. 
\item
  Si la probabilitat que un nen pateixi hemofília és 0,0001, quina és la
  probabilitat que hi hagi al menys un nen hemofílic en una escola que
  té 500 alumnes? 
\item
  La probabilitat de contreure una malaltia per contacte amb una persona
  malalta és de 2/3. Calculeu la probabilitat de contreure-la que té una
  persona sana que s'exposa a contacte successiu de dos malalts. 
\item
  La probabilitat que els cargols que fabrica una determinada empresa
  siguin defectuosos és del 10\%, però que un cargol sigui defectuós és
  independent del fet que un altre ho sigui o no. Els cargols
  s'empaqueten en capses de 5 unitats. Calculeu quina probabilitat
  tindrem que en una capsa no hi hagi cap cargol defectuós. 
\item
  El 20\% d'un model de bombetes és defectuós. En una mostra de 5
  bombetes, calculeu la probabilitat que exactament dues bombetes siguin
  defectuoses. 
\item
  El 4\% dels USB d'ordinador que fabrica una determinada empresa
  resulten defectuosos. Els USB es distribueixen en capses de 5 unitats.
  Calculeu la probabilitat que en una capsa no hi hagi cap disquet
  defectuós. 
\item
  La probabilitat que un tirador amb arc faci diana és 0,2. Si fa 5
  intents independents, calculeu la probabilitat que faci exactament 3
  dianes. 
\item
  Llancem dues monedes i anotam el nombre de cares. Calcula la mitjana i
  la desviació típica d'aquest experiment.
\item
  Considera l'experiment de llançar una moneda 3 vegades. Indica les
  següents probabilitats. A) Probabilitat que el nombre de cares sigui
  menor que 1. B) Probabilitat que el nombre de cares sigui menor o
  igual a 1. 
\item
  Calcula la probabilitat que en llançar una moneda 15 vegades el nombre
  de cares sigui menor que 5.
\item
  Escriu l'expressió (no ho calculis) de la probabilitat que en llançar
  un dau 15 vegades el nombre de cincs sigui major que 10.
\item
  En el control de qualitat de bombetes de baix consumeix d'una fàbrica
  s'ha comprovat que el 90\% són bones. Es pren una mostra de 500
  bombetes. De mitjana, quantes seran de bona qualitat? Calcula la
  mitjana, variància i desviació típica. 
\item
  En l'estudi sobre una nova medicina per a l'hepatitis C s'ha comprovat
  que produeix curacions completes en el 80\% dels casos tractats.
  S'administra a mil nous malalts, quantes curacions esperarem que es
  produeixin?
\item
  S'ha comprovat que la distribució de probabilitat del sexe d'un nounat
  és:
\end{enumerate}

\begin{longtable}[]{@{}lll@{}}
\toprule
Sexe del nounat: & noia & noi\tabularnewline
Probabilitat: & 0'485 & 0'515\tabularnewline
\bottomrule
\end{longtable}

DISTRIBUCIONS DE PROBABILITAT CONTÍNUES. LA NORMAL

\begin{enumerate}
\def\labelenumi{\arabic{enumi}.}
\item
  Calcula en una distribució normal estàndard les probabilitats
  següents: 
\item
  a) \emph{P}(\emph{z} = 0), b) \emph{P}(\emph{z} \textless{} 0), c)
  \emph{P}(\emph{z} = 1'82), d) \emph{P}(\emph{z} \textgreater{} 1'82).
\item
  Calcula en una distribució normal estàndard les probabilitats
  següents: 
\item
  a) \emph{P}(\emph{z} \textgreater{} 4), b) \emph{P}(\emph{z}
  \textless{} 4), c) \emph{P}(\emph{z} \textgreater{} 1), d)
  \emph{P}(\emph{z} \textless{} 1).
\item
  Calcula en una distribució normal estàndard les probabilitats
  següents: 
\item
  a) \emph{P}(1 \textless{} \emph{z \textless{} }2), b) \emph{P}(1'3
  \textless{} \emph{z} \textless{} 4), c) \emph{P}(0'2 \textless{}
  \emph{z} \textless{} 2'34), d) \emph{P}(1 \textless{}\emph{z}
  \textless{} 1).
\item
  Calcula en una distribució normal \emph{N}(1, 2) les probabilitats
  següents: 
\item
  a) \emph{P}(\emph{x} \textgreater{} 4), b) \emph{P}(\emph{x}
  \textless{} 4), c) \emph{P}(\emph{x} \textgreater{} 1), d)
  \emph{P}(\emph{x} \textless{} 1).
\item
  Calcula en una distribució normal \emph{N}(0'5, 0'2) les probabilitats
  següents: 
\item
  a) \emph{P}(\emph{x} \textgreater{} 4), b) \emph{P}(\emph{x}
  \textless{} 4), c) \emph{P}(\emph{x} \textgreater{} 1), d)
  \emph{P}(\emph{x} \textless{} 1).
\item
  Calcula en una distribució normal \emph{N}(1, 1/2) les probabilitats
  següents:

  a) \emph{P}(1 \textless{} \emph{x \textless{} }2), b) \emph{P}(1'3
  \textless{} \emph{x} \textless{} 4), c) \emph{P}(0'2 \textless{}
  \emph{x} \textless{} 2'34), d) \emph{P}(1 \textless{} \emph{x}
  \textless{} 3)
\item
  Les alçades de les persones d'una certa població es distribueixen
  segons una normal de mitjana 180 cm i desviació típica 15 cm.
  Determina les probabilitat que: 
\end{enumerate}

a) Una persona tingui una alçada superior a 190 cm.

b) Una persona tingui una alçada menor a 160 cm.

c) Quina proporció de persones tenen una alçada compresa entre 160 cm i
190 cm?

\begin{enumerate}
\def\labelenumi{\arabic{enumi}.}
\tightlist
\item
  En un examen per entrar en un cos de l'Estat se sap que els punts
  obtinguts es distribueixen segons una normal de mitjana 100 i
  desviació típica 10 punts. Determina la probabilitat que:
\end{enumerate}

a) Un opositor obtingui 120 punts.

b) Si per aprovar és necessari tenir més de 120 punts, Quin percentatge
d'opositors aproven?

\begin{enumerate}
\def\labelenumi{\arabic{enumi}.}
\item
\end{enumerate}

\begin{enumerate}
\def\labelenumi{\arabic{enumi}.}
\tightlist
\item
  Es llança una moneda mil vegades, quin és la probabilitat que el
  nombre de cares obtingudes estigui entre 400 i 600? I que sigui major
  que 800?
\item
  En una fàbrica de bombetes de baix consum se sap que el 70\% d'elles
  tenen una vida mitjana superior a 1000 hores. Es pren una mostra de 50
  bombetes, quin és la probabilitat que hi hagi entre 20 i 30 la vida
  mitjana de la qual sigui superior a mil hores?, i la probabilitat que
  hi hagi més de 45 la vida mitjana de la qual sigui superior a 1000
  hores?
\end{enumerate}

\begin{enumerate}
\def\labelenumi{\arabic{enumi}.}
\tightlist
\item
  Una companyia aèria ha estudiat que el 5\% de les persones que
  reserven un bitllet per a un vol no es presenten, per la qual cosa
  venen més bitllets que les places disponibles. Un determinat avió de
  la companyia té 260 places (amb el que solen reservar fins a 270).
  Calcula la probabilitat que arribin 260 passatgers. En 500 vols
  d'aquest avió, en quants consideres que hi haurà excés de passatgers? 
\item
  Es llança 600 vegades un dau i mirem el nombre de 5s. a) Quin és
  l'interval simètric respecte de la mitjana amb una probabilitat de
  0'99? b) El mateix amb una probabilitat del 0'6. 
\end{enumerate}

RESUM

\begin{longtable}[]{@{}lll@{}}
\toprule
\begin{minipage}[t]{0.32\columnwidth}\raggedright\strut
Propietats de la distribució de probabilitat discreta\strut
\end{minipage} & \begin{minipage}[t]{0.32\columnwidth}\raggedright\strut
\strut
\end{minipage} & \begin{minipage}[t]{0.32\columnwidth}\raggedright\strut
Llancem dues monedes i expliquem el nombre de cares:\strut
\end{minipage}\tabularnewline
Propietats de funció de distribució contínua &\tabularnewline
Esperança matemàtica & &  = 0(1/4) + 1(1/2) + 2(1/4) =
1\tabularnewline
\begin{minipage}[t]{0.32\columnwidth}\raggedright\strut
Variància i desviació típica\strut
\end{minipage} & \begin{minipage}[t]{0.32\columnwidth}\raggedright\strut
\strut
\end{minipage} & \begin{minipage}[t]{0.32\columnwidth}\raggedright\strut
\textsuperscript{2} = (01)\textsuperscript{2}(1/4) +
(11)\textsuperscript{2}(1/2) + (21)\textsuperscript{2}(1/4) =
1/2.\strut
\end{minipage}\tabularnewline
\begin{minipage}[t]{0.32\columnwidth}\raggedright\strut
Distribució binomial\strut
\end{minipage} & \begin{minipage}[t]{0.32\columnwidth}\raggedright\strut
\emph{E}(\emph{x}) =  = \emph{n}\emph{p},

\textsuperscript{2} =\emph{ n}\emph{pq =
n}\emph{p}(1\emph{p})\emph{ }\strut
\end{minipage} & \begin{minipage}[t]{0.32\columnwidth}\raggedright\strut
\emph{B}(10, 1/2).\strut
\end{minipage}\tabularnewline
Distribució normal & &
\includegraphics[width=5.86300cm,height=2.75200cm]{Pictures/1000000000000170000000AD88218A3D.gif}\tabularnewline
Aproximació de la binomial a la normal & Una binomial amb \emph{npq}  9
es considera s'ajusta bé a una normal d'igual mitjana i desviació
típica. &\tabularnewline
\bottomrule
\end{longtable}

\begin{longtable}[]{@{}ll@{}}
\toprule
ÀREES SOTA LA DISTRIBUCIÓ DE PROBABILITAT NORMAL ESTÀNDARD, N(0, 1)
&\tabularnewline
~\emph{\textbf{\emph{\textbf{Taula de la uam: Universitat Autònoma de
Madrid}}}} &\tabularnewline
z & 0\tabularnewline
0,0 & 0,5000\tabularnewline
0,1 & 0,5398\tabularnewline
0,2 & 0,5793\tabularnewline
0,3 & 0,6179\tabularnewline
0,4 & 0,6554\tabularnewline
0,5 & 0,6915\tabularnewline
0,6 & 0,7257\tabularnewline
0,7 & 0,7580\tabularnewline
0,8 & 0,7881\tabularnewline
0,9 & 0,8159\tabularnewline
1,0 & 0,8413\tabularnewline
1,1 & 0,8643\tabularnewline
1,2 & 0,8849\tabularnewline
1,3 & 0,9032\tabularnewline
1,4 & 0,9192\tabularnewline
1,5 & 0,9332\tabularnewline
1,6 & 0,9452\tabularnewline
1,7 & 0,9554\tabularnewline
1,8 & 0,9641\tabularnewline
1,9 & 0,9713\tabularnewline
2,0 & 0,9772\tabularnewline
2,1 & 0,9821\tabularnewline
2,2 & 0,9861\tabularnewline
2,3 & 0,9893\tabularnewline
2,4 & 0,9918\tabularnewline
2,5 & 0,9938\tabularnewline
2,6 & 0,9953\tabularnewline
2,7 & 0,9965\tabularnewline
2,8 & 0,9974\tabularnewline
2,9 & 0,9981\tabularnewline
3,0 & 0,9987\tabularnewline
3,1 & 0,9990\tabularnewline
3,2 & 0,9993\tabularnewline
3,3 & 0,9995\tabularnewline
3,4 & 0,9997\tabularnewline
3,5 & 0,9998\tabularnewline
3,6 & 0,9998\tabularnewline
3,7 & 0,9999\tabularnewline
3,8 & 0,9999\tabularnewline
3,9 & 1,0000\tabularnewline
4,0 & 1,0000\tabularnewline
\bottomrule
\end{longtable}
