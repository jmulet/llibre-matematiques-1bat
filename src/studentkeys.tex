
 \vspace{1cm} 
 
 \needspace{5\baselineskip} 
 \heading{Solucions del Tema 1}

\vspace{0.3cm}

 \needspace{3\baselineskip} 

\hyperlink{page.12}{\textbf{\em Pàgina 12}}
\begin{enumerate}


 \needspace{2\baselineskip} 

\phantomsection
 \item[\fontfamily{phv}\selectfont\color{blue}\textbf{\ref{exer:18}. }] \label{ans:18}
 \begin{tasks}[column-sep=1em, item-indent=1.3333em](3)
	 \task $\sqrt [{6}]{5} $
	 \task $\sqrt [{8}]{8} $
	 \task $\sqrt [{8}]{x^{7} } $
\end{tasks}
 \end{enumerate}
\begin{enumerate}


 \needspace{2\baselineskip} 

\phantomsection
 \item[\fontfamily{phv}\selectfont\color{blue}\textbf{\ref{exer:19}. }] \label{ans:19}
 \begin{tasks}[column-sep=1em, item-indent=1.3333em](3)
	 \task $2\,\sqrt [{}]{2} $
	 \task $9\,\sqrt [{}]{3} $
	 \task $\sqrt [{3}]{6} $
	 \task $2\,\sqrt [{4}]{2} $
	 \task $\sqrt [{3}]{3} $
	 \task $\sqrt [{3}]{49} $
	 \task $\sqrt [{8}]{7} $
	 \task $\sqrt {2} $
\end{tasks}


 \needspace{2\baselineskip} 

\phantomsection
 \item[\fontfamily{phv}\selectfont\color{blue}\textbf{\ref{exer:20}. }] \label{ans:20}
 \begin{tasks}[column-sep=1em, item-indent=1.3333em](2)
	 \task* $\sqrt [{4}]{2^{3} } =\sqrt [{12}]{2^{9} } $
	 \task* $\sqrt {7} =\sqrt [{16}]{7^{8} } $
	 \task* $\sqrt [{4}]{a^{6} } =\sqrt {a^{3} } $
	 \task* $\sqrt [{6}]{5^{12} } =\sqrt [{3}]{5^{6} } =5^{2} $
\end{tasks}
 \end{enumerate}
\vspace{0.3cm}

 \needspace{3\baselineskip} 

\hyperlink{page.13}{\textbf{\em Pàgina 13}}
\begin{enumerate}


 \needspace{2\baselineskip} 

\phantomsection
 \item[\fontfamily{phv}\selectfont\color{blue}\textbf{\ref{exer:21}. }] \label{ans:21}
 \begin{tasks}[column-sep=1em, item-indent=1.3333em](2)
	 \task $\sqrt [{3}]{5/4} $
	 \task $\sqrt [{12}]{2^{3} \cdot 3} $
	 \task $\sqrt [{12}]{5^{7} } $
	 \task $2^{3} $
	 \task $2^{4} \cdot \sqrt [{5}]{2^{4} } $
	 \task $\sqrt [{}]{5} $
	 \task 3
	 \task $5^{2} $
\end{tasks}
 \end{enumerate}
\begin{enumerate}


 \needspace{2\baselineskip} 

\phantomsection
 \item[\fontfamily{phv}\selectfont\color{blue}\textbf{\ref{exer:22}. }] \label{ans:22}
 \begin{tasks}[column-sep=1em, item-indent=1.3333em](3)
	 \task  $15\,\sqrt [{}]{11} $
	 \task $6\,\sqrt [{3}]{2} $
	 \task $-\sqrt [{}]{6} /5$
	 \task $-14\,\sqrt [{4}]{2} /3$
	 \task $41\,\sqrt [{}]{3} /15$
	 \task $13\,\sqrt [{}]{2} /5$
\end{tasks}
\phantomsection
\item[\fontfamily{phv}\selectfont\color{blue}\textbf{\ref{exer:23}. }] \label{ans:23} 
$(2+a)^{2} +4(2+a)\sqrt {a} +4a$


 \needspace{2\baselineskip} 

\phantomsection
 \item[\fontfamily{phv}\selectfont\color{blue}\textbf{\ref{exer:24}. }] \label{ans:24}
 \begin{tasks}[column-sep=1em, item-indent=1.3333em](2)
	 \task $\frac {\sqrt [{3}]{3^{2} } }{3} $
	 \task $\frac {3}{4} \,\sqrt [{4}]{2^{3} }$
	 \task $3(\sqrt {2+1} )$
	 \task $3+2\sqrt {2} $
	 \task $(\sqrt {10} -\sqrt {6} )/2$
	 \task $-(9+4\sqrt {5} )$
\end{tasks}


 \needspace{2\baselineskip} 

\phantomsection
 \item[\fontfamily{phv}\selectfont\color{blue}\textbf{\ref{exer:25}. }] \label{ans:25}
 \begin{tasks}[column-sep=1em, item-indent=1.3333em](2)
	 \task $3\sqrt {6}$
	 \task* $\frac {\sqrt [{4}]{2^{3} } }{2} $
	 \task $\frac {\sqrt {3} }{2} $
	 \task 35
	 \task $2^{-64/15} $
	 \task $33/4-5\sqrt {2} $
\end{tasks}
 \end{enumerate}
\vspace{0.3cm}

 \needspace{3\baselineskip} 

\hyperlink{page.14}{\textbf{\em Pàgina 14}}
\begin{enumerate}
\phantomsection
\item[\fontfamily{phv}\selectfont\color{blue}\textbf{\ref{exer:27}. }] \label{ans:27} 
$x=-10$ i $x=4$
 \end{enumerate}
\begin{enumerate}
\phantomsection
\item[\fontfamily{phv}\selectfont\color{blue}\textbf{\ref{exer:28}. }] \label{ans:28} 
$\sqrt [8]{x^3}$
\phantomsection
\item[\fontfamily{phv}\selectfont\color{blue}\textbf{\ref{exer:29}. }] \label{ans:29} 
$13+4\sqrt {3}$
\phantomsection
\item[\fontfamily{phv}\selectfont\color{blue}\textbf{\ref{exer:30}. }] \label{ans:30} 
$\dfrac {4\sqrt {5}}{5}$
\phantomsection
\item[\fontfamily{phv}\selectfont\color{blue}\textbf{\ref{exer:31}. }] \label{ans:31} 
$\frac {13}{3}\sqrt [3]{2}$
\phantomsection
\item[\fontfamily{phv}\selectfont\color{blue}\textbf{\ref{exer:32}. }] \label{ans:32} 
$\dfrac {16-5\sqrt {15}}{-7}$
\phantomsection
\item[\fontfamily{phv}\selectfont\color{blue}\textbf{\ref{exer:33}. }] \label{ans:33} 
$(-1,\,6]$
 \end{enumerate}

 \vspace{1cm} 
 
 \needspace{5\baselineskip} 
 \heading{Solucions del Tema 2}

\vspace{0.3cm}

 \needspace{3\baselineskip} 

\hyperlink{page.17}{\textbf{\em Pàgina 17}}
\begin{enumerate}


 \needspace{2\baselineskip} 

\phantomsection
 \item[\fontfamily{phv}\selectfont\color{blue}\textbf{\ref{exer:37}. }] \label{ans:37}
 \begin{tasks}[column-sep=1em, item-indent=1.3333em](1)
	 \task*  $Q(x)=3x^{3}+4x^{2}-x+2$; $R(x)=-4$
	 \task $Q(x)=2x+2$; $R(x)=2x-1$
	 \task* $Q(x)=a(x^{3}+ x^{2}+ x+ 1)$; $R(x)=a+b$
	 \task* $Q(x)=x^{8}+ x^{6}+ x^{4}+ x^{2}+ 1$; $R(x)=0$ 
\end{tasks}
 \end{enumerate}
\vspace{0.3cm}

 \needspace{3\baselineskip} 

\hyperlink{page.19}{\textbf{\em Pàgina 19}}
\begin{enumerate}


 \needspace{2\baselineskip} 

\phantomsection
 \item[\fontfamily{phv}\selectfont\color{blue}\textbf{\ref{exer:45}. }] \label{ans:45}
 \begin{tasks}[column-sep=1em, item-indent=1.3333em](1)
	 \task  $3(x+2)\cdot (x+1)$
	 \task $x^{3}\cdot (x+3)\cdot (x-3)$
	 \task $4(x+3)\cdot (x+1)\cdot (x-1)$
	 \task $-2(x+1)^{2}\cdot (x-3)$
	 \task $x\cdot (x^{3}-x^{2}+8x-4)$
	 \task ($x+1)^{2}\cdot (x-2)$
	 \task $2(x+1)\cdot (x-2)\cdot (x-5)$
	 \task $x^{2}\cdot (x+1)\cdot (x-3)$
	 \task $(x+4)\cdot (x+1)\cdot (x-2)$ 
\end{tasks}
 \end{enumerate}
\vspace{0.3cm}

 \needspace{3\baselineskip} 

\hyperlink{page.20}{\textbf{\em Pàgina 20}}
\begin{enumerate}


 \needspace{2\baselineskip} 

\phantomsection
 \item[\fontfamily{phv}\selectfont\color{blue}\textbf{\ref{exer:50}. }] \label{ans:50}
 \begin{tasks}[column-sep=1em, item-indent=1.3333em](1)
	 \task  $\dfrac {x-1}{3x(x+2)} $
	 \task $\dfrac {2(x+5)}{(x+1)^{2} } $
	 \task $\dfrac {x-1}{x(x+2)} $
	 \task \textit {No es pot}
	 \task* $\dfrac {(x^{2} +x+1)(x-1)}{x^{} } $
	 \task $\dfrac {1}{x+2} $
	 \task $\dfrac {x+2}{x-1} $
	 \task* $\dfrac {2(x^{4} -x^{3} +x^{2} -x+1)}{x} $
\end{tasks}
 \end{enumerate}
\vspace{0.3cm}

 \needspace{3\baselineskip} 

\hyperlink{page.21}{\textbf{\em Pàgina 21}}
\begin{enumerate}


 \needspace{2\baselineskip} 

\phantomsection
 \item[\fontfamily{phv}\selectfont\color{blue}\textbf{\ref{exer:54}. }] \label{ans:54}
 \begin{tasks}[column-sep=1em, item-indent=1.3333em](2)
	 \task  $\dfrac {-4x}{(x+1)(x-1)} $
	 \task $\dfrac {-2}{x+1} $
	 \task $\dfrac {-1}{x-1} $
	 \task $\dfrac {-2t+3}{t(t+2)} $
	 \task 0
	 \task $\dfrac {1-x^{2} }{x^{2} } $
	 \task $\dfrac {3x^{2} +5}{x(x+1)^{2} } $
\end{tasks}
 \end{enumerate}
\vspace{0.3cm}

 \needspace{3\baselineskip} 

\hyperlink{page.22}{\textbf{\em Pàgina 22}}
\begin{enumerate}


 \needspace{2\baselineskip} 

\phantomsection
 \item[\fontfamily{phv}\selectfont\color{blue}\textbf{\ref{exer:58}. }] \label{ans:58}
 \begin{tasks}[column-sep=1em, item-indent=1.3333em](2)
	 \task  $x=1,\,2,\,-2$
	 \task $x=0,\,5,\,-5$
	 \task $x=1,\,-2$
	 \task $x=\pm 2,\,3,\,-1$
	 \task $x=1,\,3,\,5,\,-4$
	 \task $x=1$
	 \task $x=-2,\,-1,\,2$
	 \task $x=-3,\,-1,\,2$
	 \task $x=-2,\,2,\,4$
	 \task $x=-3,\,-2,\,1$
	 \task $x=-3,\,3,\,-2,\,2$
	 \task $x=-1,\,0,\,5$ 
\end{tasks}
 \end{enumerate}
\vspace{0.3cm}

 \needspace{3\baselineskip} 

\hyperlink{page.23}{\textbf{\em Pàgina 23}}
\begin{enumerate}


 \needspace{2\baselineskip} 

\phantomsection
 \item[\fontfamily{phv}\selectfont\color{blue}\textbf{\ref{exer:64}. }] \label{ans:64}
 \begin{tasks}[column-sep=1em, item-indent=1.3333em](2)
	 \task  $x= \dfrac {2}{3},\,-\dfrac {1}{2}$
	 \task $x=2,\,\dfrac {1}{7}$
	 \task $x=2,\,-\dfrac {3}{5}$
	 \task $\dfrac {1\pm \sqrt {41}}{4}$
\end{tasks}
 \end{enumerate}
\begin{enumerate}


 \needspace{2\baselineskip} 

\phantomsection
 \item[\fontfamily{phv}\selectfont\color{blue}\textbf{\ref{exer:66}. }] \label{ans:66}
 \begin{tasks}[column-sep=1em, item-indent=1.3333em](2)
	 \task  $x=4$
	 \task $x=4$
	 \task $x=9$
	 \task $x=7$
	 \task $x=2$
	 \task $x=38414$
	 \task $x=10$
	 \task $x=3$
	 \task $x=11$
	 \task $x=29$
	 \task $x=14$
	 \task $x=1$
\end{tasks}
 \end{enumerate}
\vspace{0.3cm}

 \needspace{3\baselineskip} 

\hyperlink{page.24}{\textbf{\em Pàgina 24}}
\begin{enumerate}


 \needspace{2\baselineskip} 

\phantomsection
 \item[\fontfamily{phv}\selectfont\color{blue}\textbf{\ref{exer:67}. }] \label{ans:67}
 \begin{tasks}[column-sep=1em, item-indent=1.3333em](2)
	 \task  $x=1; y=16$
	 \task $x=6; y=8$
	 \task $x=10; y=2$
	 \task $x=4; y=7$
	 \task $x=3; y=1$
	 \task $x=-2; y=8$ 
\end{tasks}
 \end{enumerate}
\begin{enumerate}


 \needspace{2\baselineskip} 

\phantomsection
 \item[\fontfamily{phv}\selectfont\color{blue}\textbf{\ref{exer:69}. }] \label{ans:69}
 \begin{tasks}[column-sep=1em, item-indent=1.3333em](1)
	 \task  $x=7; y=2; z=11$
	 \task $x=4; y=-3; z=0$
	 \task $x=-1; y=4; z=4$
	 \task $x=8; y=4; z=-3$ 
\end{tasks}


 \needspace{2\baselineskip} 

\phantomsection
 \item[\fontfamily{phv}\selectfont\color{blue}\textbf{\ref{exer:70}. }] \label{ans:70}
 \begin{tasks}[column-sep=1em, item-indent=1.3333em](1)
	 \task  $x=1; y=-5; z=4$
	 \task $x=-1; y=-2; z=-2$
	 \task $x=15; y=2; z=1$
	 \task $x=3; y=4; z=9$ 
\end{tasks}
 \end{enumerate}
\vspace{0.3cm}

 \needspace{3\baselineskip} 

\hyperlink{page.25}{\textbf{\em Pàgina 25}}
\begin{enumerate}


 \needspace{2\baselineskip} 

\phantomsection
 \item[\fontfamily{phv}\selectfont\color{blue}\textbf{\ref{exer:71}. }] \label{ans:71}
 \begin{tasks}[column-sep=1em, item-indent=1.3333em](1)
	 \task  $x=1; y=-2; z=3$
	 \task $x=4; y=2; z=-3$
	 \task $x=1; y=-1; z=0$
	 \task $x=2; y=\dfrac {1}{5}; z=-1$ 
\end{tasks}
 \end{enumerate}
\vspace{0.3cm}

 \needspace{3\baselineskip} 

\hyperlink{page.28}{\textbf{\em Pàgina 28}}
\begin{enumerate}
\phantomsection
\item[\fontfamily{phv}\selectfont\color{blue}\textbf{\ref{exer:89}. }] \label{ans:89} 
$-3$
 \end{enumerate}
\begin{enumerate}
\phantomsection
\item[\fontfamily{phv}\selectfont\color{blue}\textbf{\ref{exer:90}. }] \label{ans:90} 
$Q=x^3$; $R=1$
\phantomsection
\item[\fontfamily{phv}\selectfont\color{blue}\textbf{\ref{exer:91}. }] \label{ans:91} 
No, si és de grau 4 pot tenir 4 arrels, que poden esser 4 reals, o be 2 arrels reals i 2 complexes o totes 4 complexes.
\phantomsection
\item[\fontfamily{phv}\selectfont\color{blue}\textbf{\ref{exer:92}. }] \label{ans:92} 
$x \in [-2, 2]$
\phantomsection
\item[\fontfamily{phv}\selectfont\color{blue}\textbf{\ref{exer:93}. }] \label{ans:93} 
$[-1, 15]$
\phantomsection
\item[\fontfamily{phv}\selectfont\color{blue}\textbf{\ref{exer:94}. }] \label{ans:94} 
$x \geq 9/5$ 
\phantomsection
\item[\fontfamily{phv}\selectfont\color{blue}\textbf{\ref{exer:95}. }] \label{ans:95} 
$x\in (1,2)$


 \needspace{2\baselineskip} 

\phantomsection
 \item[\fontfamily{phv}\selectfont\color{blue}\textbf{\ref{exer:96}. }] \label{ans:96}
 \begin{tasks}[column-sep=1em, item-indent=1.3333em](4)
	 \task F
	 \task V
	 \task F
	 \task F
\end{tasks}
 \end{enumerate}

 \vspace{1cm} 
 
 \needspace{5\baselineskip} 
 \heading{Solucions del Tema 3}

\vspace{0.3cm}

 \needspace{3\baselineskip} 

\hyperlink{page.33}{\textbf{\em Pàgina 33}}
\begin{enumerate}


 \needspace{2\baselineskip} 

\phantomsection
 \item[\fontfamily{phv}\selectfont\color{blue}\textbf{\ref{exer:110}. }] \label{ans:110}
 \begin{tasks}[column-sep=1em, item-indent=1.3333em](1)
	 \task* $\hat C=33$; $b=26,8$; $c=17,4$
	 \task* $\hat B=67$; $b=66,3$; $c=28,1$
	 \task* $\hat C=39$; $\hat B=51$; $a=396,7$
	 \task* $\hat B=58$; $b=56,01$; $a=66,05$
\end{tasks}
 \end{enumerate}
\begin{enumerate}
\phantomsection
\item[\fontfamily{phv}\selectfont\color{blue}\textbf{\ref{exer:111}. }] \label{ans:111} 
$a=3,46$; $b=1,73$
\phantomsection
\item[\fontfamily{phv}\selectfont\color{blue}\textbf{\ref{exer:112}. }] \label{ans:112} 
$\alpha =25,5^\circ $
\phantomsection
\item[\fontfamily{phv}\selectfont\color{blue}\textbf{\ref{exer:113}. }] \label{ans:113} 
 $a=25$; $c=20$; $\hat B=36,87^\circ $; $\hat C=53,13^\circ $
 \end{enumerate}
\vspace{0.3cm}

 \needspace{3\baselineskip} 

\hyperlink{page.37}{\textbf{\em Pàgina 37}}
\begin{enumerate}
\phantomsection
\item[\fontfamily{phv}\selectfont\color{blue}\textbf{\ref{exer:140}. }] \label{ans:140} 
$d=35,49$ m
 \end{enumerate}
\vspace{0.3cm}

 \needspace{3\baselineskip} 

\hyperlink{page.38}{\textbf{\em Pàgina 38}}
\begin{enumerate}
\phantomsection
\item[\fontfamily{phv}\selectfont\color{blue}\textbf{\ref{exer:141}. }] \label{ans:141} 
Del triangle $\widehat {CAD}$ troba $\overline {AD}=74.16$ km, del triangle $\widehat {CBD}$ troba $\overline {BD}=52.05$ km pel teorema del sinus i finalment del triangle $\widehat {ADB}$ troba $\overline {AB}=24$ km pel teorema del cosinus.
 \end{enumerate}
\begin{enumerate}
\phantomsection
\item[\fontfamily{phv}\selectfont\color{blue}\textbf{\ref{exer:142}. }] \label{ans:142} 
cim A=827 m, cim B=751 m, distància entre cims AB=1687.2 m
 \end{enumerate}
\vspace{0.3cm}

 \needspace{3\baselineskip} 

\hyperlink{page.39}{\textbf{\em Pàgina 39}}
\begin{enumerate}
\phantomsection
\item[\fontfamily{phv}\selectfont\color{blue}\textbf{\ref{exer:150}. }] \label{ans:150} 
Treu factor comú $\cos \alpha $ i utilitza la relació fonamental.
 \end{enumerate}
\begin{enumerate}
\phantomsection
\item[\fontfamily{phv}\selectfont\color{blue}\textbf{\ref{exer:151}. }] \label{ans:151} 
Desenvolupa el quadrat amb la identitat notable $(a+b)^2 = a^2 + b^2 + 2ab$. Empra la relació fonamental i la fórmula de $\sin 2\alf $.
\phantomsection
\item[\fontfamily{phv}\selectfont\color{blue}\textbf{\ref{exer:152}. }] \label{ans:152} 
Utilitza les relacions de l'angle oposat $\cos (-x)=\cos x$, $\sin (-x)=-\sin x$, $\tg (-x)=-\tg x$.
\phantomsection
\item[\fontfamily{phv}\selectfont\color{blue}\textbf{\ref{exer:153}. }] \label{ans:153} 
Expressa $\tg \alf $ i $\cotg \alf $ com a quocients de sinus i cosinus. Després realitza la suma de fraccions amb el mínim comú múltiple. Finalment utilitza la relació fonamental.
\phantomsection
\item[\fontfamily{phv}\selectfont\color{blue}\textbf{\ref{exer:154}. }] \label{ans:154} 
Escriu $\sin (3\alf ) = \sin (\alf +2\alf )$. Aplica la fórmula de la suma d'angles i tot seguit les fórmules de l'angle doble. Finalment opera, simplifica i treu factor comú $\sin \alf $.
\phantomsection
\item[\fontfamily{phv}\selectfont\color{blue}\textbf{\ref{exer:155}. }] \label{ans:155} 
Escriu $\cos (4\alf ) = \cos (2\alf +2\alf )$. Aplica la fórmula de la suma d'angles i tot seguit les fórmules de l'angle doble. Finalment opera i simplifica.
 \end{enumerate}
\vspace{0.3cm}

 \needspace{3\baselineskip} 

\hyperlink{page.42}{\textbf{\em Pàgina 42}}
\begin{enumerate}


 \needspace{2\baselineskip} 

\phantomsection
 \item[\fontfamily{phv}\selectfont\color{blue}\textbf{\ref{exer:172}. }] \label{ans:172}
 \begin{tasks}[column-sep=1em, item-indent=1.3333em](1)
	 \task $\sin (-750^\circ )=1/2$
	 \task* $\tg 570^\circ =\frac {-1}{\sqrt 3}=-\frac {\sqrt 3}{3}$
	 \task $\cos 20\pi /3 = -1/2$
\end{tasks}
 \end{enumerate}
\begin{enumerate}
\phantomsection
\item[\fontfamily{phv}\selectfont\color{blue}\textbf{\ref{exer:173}. }] \label{ans:173} 
$\sin (105)=\sin (60+45)=\frac {\sqrt {6}+\sqrt {2}}{4}$ \par $\cos (75)=\sin (30+45)=\frac {\sqrt {6}-\sqrt {2}}{4}$
\phantomsection
\item[\fontfamily{phv}\selectfont\color{blue}\textbf{\ref{exer:174}. }] \label{ans:174} 
$c=17,32$, $\hat B=90^\circ $, $\hat C=60^\circ $
\phantomsection
\item[\fontfamily{phv}\selectfont\color{blue}\textbf{\ref{exer:175}. }] \label{ans:175} 
Sí ho aconseguirà. Estan a 105,83 m.
\phantomsection
\item[\fontfamily{phv}\selectfont\color{blue}\textbf{\ref{exer:176}. }] \label{ans:176} 
$\sin a=\frac {-1}{\sqrt 5}=-\frac {\sqrt 5}{5}$\par $\cos a=\frac {-2}{\sqrt 5}=-\frac {2\sqrt 5}{5}$;
\phantomsection
\item[\fontfamily{phv}\selectfont\color{blue}\textbf{\ref{exer:177}. }] \label{ans:177} 
a) $x=\left \{\begin {array}{l} 120 + n 360 \\ 240 +n 360 \end {array}\right .$ \par b) $x=45 + n 180$
\phantomsection
\item[\fontfamily{phv}\selectfont\color{blue}\textbf{\ref{exer:178}. }] \label{ans:178} 
a) $(60+360k,120–360k)$ i \par $(120+360k,60–360k)$; \par b) $(75+360k, 15–360k)$ i \par $(15+360k, 75–360k)$
\phantomsection
\item[\fontfamily{phv}\selectfont\color{blue}\textbf{\ref{exer:179}. }] \label{ans:179} 
Substituir el $\sin (2a)$ pel seu valor i aplicar que $1–\cos 2a$ és el sinus al quadrat.
\phantomsection
\item[\fontfamily{phv}\selectfont\color{blue}\textbf{\ref{exer:180}. }] \label{ans:180} 
Perímetre = $5 \cdot 35.267 = 176,3355$ m; Apotema $a_p = 30 \cos 36 =24.27$; \quad Àrea = 2139.83 m$^2$.
\phantomsection
\item[\fontfamily{phv}\selectfont\color{blue}\textbf{\ref{exer:181}. }] \label{ans:181} 
$8,63^\circ $
 \end{enumerate}

 \vspace{1cm} 
 
 \needspace{5\baselineskip} 
 \heading{Solucions del Tema 4}

\vspace{0.3cm}

 \needspace{3\baselineskip} 

\hyperlink{page.48}{\textbf{\em Pàgina 48}}
\begin{enumerate}


 \needspace{2\baselineskip} 

\phantomsection
 \item[\fontfamily{phv}\selectfont\color{blue}\textbf{\ref{exer:183}. }] \label{ans:183}
 \begin{tasks}[column-sep=1em, item-indent=1.3333em](2)
	 \task  $5 -9i$
	 \task $-6-15i$
	 \task $-13+11i$
	 \task 0
	 \task 13
	 \task $4 + 11i$
	 \task $2i$
	 \task $-4$
\end{tasks}
 \end{enumerate}
\begin{enumerate}


 \needspace{2\baselineskip} 

\phantomsection
 \item[\fontfamily{phv}\selectfont\color{blue}\textbf{\ref{exer:184}. }] \label{ans:184}
 \begin{tasks}[column-sep=1em, item-indent=1.3333em](2)
	 \task  $-{{1}\over {10}}-{{7\,i}\over {10}}$
	 \task $-2$
	 \task ${{2}\over {5}}-{{i}\over {5}}$
	 \task ${{34\,i}\over {5}}$
\end{tasks}


 \needspace{2\baselineskip} 

\phantomsection
 \item[\fontfamily{phv}\selectfont\color{blue}\textbf{\ref{exer:187}. }] \label{ans:187}
 \begin{tasks}[column-sep=1em, item-indent=1.3333em](1)
	 \task  $|z|=2$; $arg(z)=-30^\circ $
	 \task* $|z|=\sqrt {8}$; $arg(z)=225 = - 135^\circ $
	 \task $|z|=2$; $arg(z)=-60^\circ $
	 \task $|z|=4$; $arg(z)=-90^\circ $
\end{tasks}
 \end{enumerate}
\vspace{0.3cm}

 \needspace{3\baselineskip} 

\hyperlink{page.49}{\textbf{\em Pàgina 49}}
\begin{enumerate}


 \needspace{2\baselineskip} 

\phantomsection
 \item[\fontfamily{phv}\selectfont\color{blue}\textbf{\ref{exer:192}. }] \label{ans:192}
 \begin{tasks}[column-sep=1em, item-indent=1.3333em](1)
	 \task*  $2_{\,60^\circ }= 1 + \sqrt {3} i$
	 \task* $3_{\,-45^\circ }= \frac {3\sqrt {2}}{2} - \frac {3\sqrt {2}}{2} i$
	 \task $1_{\,90^\circ }= i $
	 \task* $5_{\,120^\circ }= -\frac {5}{2} + i \frac {5\sqrt {3}}{2}$
\end{tasks}
 \end{enumerate}
\vspace{0.3cm}

 \needspace{3\baselineskip} 

\hyperlink{page.52}{\textbf{\em Pàgina 52}}
\begin{enumerate}


 \needspace{2\baselineskip} 

\phantomsection
 \item[\fontfamily{phv}\selectfont\color{blue}\textbf{\ref{exer:206}. }] \label{ans:206}
 \begin{tasks}[column-sep=1em, item-indent=1.3333em](2)
	 \task $-5-6i$
	 \task $-8-6i$
	 \task $10i$
	 \task $-2+i$
\end{tasks}
 \end{enumerate}
\begin{enumerate}
\phantomsection
\item[\fontfamily{phv}\selectfont\color{blue}\textbf{\ref{exer:207}. }] \label{ans:207} 
$-46+63i$
\phantomsection
\item[\fontfamily{phv}\selectfont\color{blue}\textbf{\ref{exer:208}. }] \label{ans:208} 
$z_1=5+2i$, $z_2=5-2i$


 \needspace{2\baselineskip} 

\phantomsection
 \item[\fontfamily{phv}\selectfont\color{blue}\textbf{\ref{exer:209}. }] \label{ans:209}
 \begin{tasks}[column-sep=1em, item-indent=1.3333em](2)
	 \task Real $k=-2$
	 \task Imaginari pur $k=-2$
\end{tasks}
\phantomsection
\item[\fontfamily{phv}\selectfont\color{blue}\textbf{\ref{exer:210}. }] \label{ans:210} 
$x=\pm 2$
\phantomsection
\item[\fontfamily{phv}\selectfont\color{blue}\textbf{\ref{exer:211}. }] \label{ans:211} 
$3\sqrt {2}$, $3\pi /4$
\phantomsection
\item[\fontfamily{phv}\selectfont\color{blue}\textbf{\ref{exer:212}. }] \label{ans:212} 
$1+\sqrt {3}i$
\phantomsection
\item[\fontfamily{phv}\selectfont\color{blue}\textbf{\ref{exer:213}. }] \label{ans:213} 
$-8i$
\phantomsection
\item[\fontfamily{phv}\selectfont\color{blue}\textbf{\ref{exer:214}. }] \label{ans:214} 
$5(\cos (\pi /2) +i \sin (\pi /2)$
\phantomsection
\item[\fontfamily{phv}\selectfont\color{blue}\textbf{\ref{exer:215}. }] \label{ans:215} 
$z_1=2_{110^\circ }$, $z_2=2_{230^\circ }$, $z_3=2_{350^\circ }$
 \end{enumerate}

 \vspace{1cm} 

 \needspace{5\baselineskip} 
 \heading{Solucions del Bloc I}

\vspace{0.3cm}

 \needspace{3\baselineskip} 

\hyperlink{page.54}{\textbf{\em Pàgina 54}}
\begin{enumerate}
\phantomsection
\item[\fontfamily{phv}\selectfont\color{blue}\textbf{\ref{exer:216}. }] \label{ans:216} 
a) $a\sqrt {a}$ \quad \quad b) $\frac {3\sqrt {2}-4\sqrt {6}}{6}$
 \end{enumerate}
\begin{enumerate}
\phantomsection
\item[\fontfamily{phv}\selectfont\color{blue}\textbf{\ref{exer:217}. }] \label{ans:217} 
a) $a\,\sqrt [20]{a}$ \quad \quad b)$\frac {2\sqrt {3}+3\sqrt {2}-6}{6}$
\phantomsection
\item[\fontfamily{phv}\selectfont\color{blue}\textbf{\ref{exer:218}. }] \label{ans:218} 
$x^2 (x+1) (x-2) (3x-1)$
\phantomsection
\item[\fontfamily{phv}\selectfont\color{blue}\textbf{\ref{exer:219}. }] \label{ans:219} 
2
\phantomsection
\item[\fontfamily{phv}\selectfont\color{blue}\textbf{\ref{exer:220}. }] \label{ans:220} 
$x=3$ vàlida.
\phantomsection
\item[\fontfamily{phv}\selectfont\color{blue}\textbf{\ref{exer:221}. }] \label{ans:221} 
$x=1$
\phantomsection
\item[\fontfamily{phv}\selectfont\color{blue}\textbf{\ref{exer:222}. }] \label{ans:222} 
$(3, +\infty )$
\phantomsection
\item[\fontfamily{phv}\selectfont\color{blue}\textbf{\ref{exer:223}. }] \label{ans:223} 
$x=38$: be, $y=18$: malament, $z=4$: no contestades. Planteig: $x+y+z=60$; $5x-2y-z=150$; $y+5z=x$
\phantomsection
\item[\fontfamily{phv}\selectfont\color{blue}\textbf{\ref{exer:224}. }] \label{ans:224} 
S.C.I. $x=1$, $y=z-3$, $z=z$
 \end{enumerate}
\vspace{0.3cm}

 \needspace{3\baselineskip} 

\hyperlink{page.55}{\textbf{\em Pàgina 55}}
\begin{enumerate}
\phantomsection
\item[\fontfamily{phv}\selectfont\color{blue}\textbf{\ref{exer:225}. }] \label{ans:225} 
Els costats són 10.49 cm i 20.25 cm, el perímetre 61.48 cm. L'àrea 83.23 cm$^2$.
 \end{enumerate}
\begin{enumerate}
\phantomsection
\item[\fontfamily{phv}\selectfont\color{blue}\textbf{\ref{exer:226}. }] \label{ans:226} 
Primer obtenim els costats resolent un sistema d'equacions $a=19$, $b=16$ i $c=13$. Després aplicam el Teorema del Cosinus per obtenir els angles $\hat A=42.54^\circ $, $\hat B=56.3^\circ $, $\hat C=81.17^\circ $
\phantomsection
\item[\fontfamily{phv}\selectfont\color{blue}\textbf{\ref{exer:227}. }] \label{ans:227} 
Utilitzam el Teorema del Sinus. $\hat C= 56^\circ \, 19^\prime \, 31^{\prime \prime }$, $\hat B= 51^\circ \, 40^\prime \, 29^{\prime \prime }$ i $\bar {AC}=263.96$ m.
\phantomsection
\item[\fontfamily{phv}\selectfont\color{blue}\textbf{\ref{exer:228}. }] \label{ans:228} 
$d=3557$ km sobre la superfície de la Terra.
\phantomsection
\item[\fontfamily{phv}\selectfont\color{blue}\textbf{\ref{exer:229}. }] \label{ans:229} 
No existeix cap angle amb aquestes condicions. S'obtindria que $\cos a = 3/4$ i amb aquesta dada no es compleix que $\sin ^2 a +\cos ^2 a = 1$.
\phantomsection
\item[\fontfamily{phv}\selectfont\color{blue}\textbf{\ref{exer:230}. }] \label{ans:230} 
En primer lloc trobam que $\sin \alpha = \frac {2\sqrt 5}{5}$ i $\cos \alpha = \frac {\sqrt 5}{5}$. a) $-3/5$ \quad b) $\frac {\sqrt 5}{5}$ \quad c) $\sqrt {\frac {5-\sqrt 5}{10}}$ \quad d) $-3$
\phantomsection
\item[\fontfamily{phv}\selectfont\color{blue}\textbf{\ref{exer:231}. }] \label{ans:231} 
a) Utilitza que $sin^4 x = (1-\cos ^2 x)^2$ desenvolupa el quadrat i simplifica. b) Utilitza que $\cos ^2 \frac {\beta }{2} = \frac {1+\cos \beta }{2}$ 
\phantomsection
\item[\fontfamily{phv}\selectfont\color{blue}\textbf{\ref{exer:232}. }] \label{ans:232} 
a) $x= 0^\circ + n\cdot 360^\circ $, $x=126.87^\circ + n\cdot 360^\circ $. b) $x= 90^\circ + n\cdot 180^\circ $, $x= 60^\circ + n\cdot 360^\circ $, $x= 300^\circ + n\cdot 360^\circ $.
\phantomsection
\item[\fontfamily{phv}\selectfont\color{blue}\textbf{\ref{exer:233}. }] \label{ans:233} 
$x=30^\circ + n\cdot 180^\circ $, $y=30^\circ + k\cdot 180^\circ $.
\phantomsection
\item[\fontfamily{phv}\selectfont\color{blue}\textbf{\ref{exer:234}. }] \label{ans:234} 
$-\frac {19}{10}+\frac {37}{10}i$
\phantomsection
\item[\fontfamily{phv}\selectfont\color{blue}\textbf{\ref{exer:235}. }] \label{ans:235} 
$i$
\phantomsection
\item[\fontfamily{phv}\selectfont\color{blue}\textbf{\ref{exer:236}. }] \label{ans:236} 
$z=5+2i$, $z=5-2i$
\phantomsection
\item[\fontfamily{phv}\selectfont\color{blue}\textbf{\ref{exer:237}. }] \label{ans:237} 
 $z^*=3_{\, 300^\circ }$, $1/z=(1/3)_{\, 300^\circ }$, $z^2=9_{\, 120^\circ }$, $\sqrt [3]{z}$ té tres resultats = $(\sqrt {3})_{\, 20^\circ }$, $(\sqrt {3})_{\, 140^\circ }$ i $(\sqrt {3})_{\, 260^\circ }$.
 \end{enumerate}

 \vspace{1cm} 
 
 \needspace{5\baselineskip} 
 \heading{Solucions del Tema 5}

\vspace{0.3cm}

 \needspace{3\baselineskip} 

\hyperlink{page.60}{\textbf{\em Pàgina 60}}
\begin{enumerate}
\phantomsection
\item[\fontfamily{phv}\selectfont\color{blue}\textbf{\ref{exer:239}. }] \label{ans:239} 
2, 3, 5, 8, 13, 21, 34, ...
 \end{enumerate}
\begin{enumerate}
\phantomsection
\item[\fontfamily{phv}\selectfont\color{blue}\textbf{\ref{exer:240}. }] \label{ans:240} 
a) $a_n=3+5(n-1)$ o $a_1=3$ $a_n=a_{n-1}+3$ \par b) $a_n=n^3$ \par c) $a_n=8\cdot (1/2)^{n-1}$ o $a_1=8$ $a_n=a_{n-1}/2$]
\phantomsection
\item[\fontfamily{phv}\selectfont\color{blue}\textbf{\ref{exer:241}. }] \label{ans:241} 
$a_n=10-3(n-1)$ i $a_{100}=-197$
\phantomsection
\item[\fontfamily{phv}\selectfont\color{blue}\textbf{\ref{exer:242}. }] \label{ans:242} 
$d=(19-11)/2=4$ i $a_1=3$, \linebreak $a_n=3+4(n-1)$, $S_{100}=20100$
\phantomsection
\item[\fontfamily{phv}\selectfont\color{blue}\textbf{\ref{exer:243}. }] \label{ans:243} 
$a_n=100\cdot (0.5)^{n-1}$, $a_{50}=1.776\cdot 10^{-13}$, $S_\infty =200$
\phantomsection
\item[\fontfamily{phv}\selectfont\color{blue}\textbf{\ref{exer:244}. }] \label{ans:244} 
$r=3$ i $a_1=1$, $a_n=3^{n-1}$, $S_{30}=1.029\cdot 10^{14}$
 \end{enumerate}
\vspace{0.3cm}

 \needspace{3\baselineskip} 

\hyperlink{page.62}{\textbf{\em Pàgina 62}}
\begin{enumerate}
\phantomsection
\item[\fontfamily{phv}\selectfont\color{blue}\textbf{\ref{exer:247}. }] \label{ans:247} 
$\text {Dom }f=\Re -\{\pm 2\}$,\par $\text {Dom } g=(-\infty ,-2/3]\cup (3,+\infty )$,\par $\text {Dom } h=\Re -\{1\}$,\par $\text {Dom } i=\Re -\{\pm 1\}$,\par $\text {Dom } j=(-\infty ,-3]\cup (3,+\infty )$,\par $\text {Dom } k=\Re -\{\pm \sqrt {3}\}$,\par $\text {Dom } l=[-2,3)$, $\text {Dom } m=\Re - \{1\}$
 \end{enumerate}
\vspace{0.3cm}

 \needspace{3\baselineskip} 

\hyperlink{page.66}{\textbf{\em Pàgina 66}}
\begin{enumerate}
\phantomsection
\item[\fontfamily{phv}\selectfont\color{blue}\textbf{\ref{exer:271}. }] \label{ans:271} 
$p^{-1}=(3-x)/5$,\par $q^{-1}=\pm \sqrt {(x+1)/2}$,\par $r^{-1}=\sqrt [3]{6-x}$, $s^{-1}=-x$,\par $f^{-1}=(3x+4)/(2-x)$, $g^{-1}=-3/x$,\par $h^{-1}=(1\pm \sqrt {1+4x})/2$,\par $j^{-1}=\pm \sqrt {4x/(1+x)}$, \,\,$k^{-1}=4+\ln x$,\par $l^{-1}=1/\log _2 {x}$,\par $m^{-1}=\log x /(\log 2 - \log 3)$,\par $n^{-1}=\ln x / (\ln x -1)$,\par $a^{-1}=2+e^x$, $b^{-1}=3\cdot 10^x +1$,\par $c^{-1}=(1+4 e^x)/(1-2 e^x)$,\par $d^{-1}=\sqrt [3]{1+10^x}$ 
 \end{enumerate}
\vspace{0.3cm}

 \needspace{3\baselineskip} 

\hyperlink{page.67}{\textbf{\em Pàgina 67}}
\begin{enumerate}


 \needspace{2\baselineskip} 

\phantomsection
 \item[\fontfamily{phv}\selectfont\color{blue}\textbf{\ref{exer:275}. }] \label{ans:275}
 \begin{tasks}[column-sep=1em, item-indent=1.3333em](1)
	 \task $x=15$ ($x=0$ no vàlida)
	 \task $x=2$
	 \task $x=100$ ($x=0$ no vàlida)
	 \task $x=2/(\log 2 - 1)$
	 \task $x=12/5$
	 \task $x=1$ i $x=-2$
\end{tasks}
 \end{enumerate}
\vspace{0.3cm}

 \needspace{3\baselineskip} 

\hyperlink{page.69}{\textbf{\em Pàgina 69}}
\begin{enumerate}
\phantomsection
\item[\fontfamily{phv}\selectfont\color{blue}\textbf{\ref{exer:286}. }] \label{ans:286} 
\textbf {-- 10.} Autoavaluació: 1a; 2d; 3d; 4b; 5c; 6b; 7b; 8a; 9c; 10c
 \end{enumerate}

 \vspace{1cm} 
 
 \needspace{5\baselineskip} 
 \heading{Solucions del Tema 6}

\vspace{0.3cm}

 \needspace{3\baselineskip} 

\hyperlink{page.74}{\textbf{\em Pàgina 74}}
\begin{enumerate}


 \needspace{2\baselineskip} 

\phantomsection
 \item[\fontfamily{phv}\selectfont\color{blue}\textbf{\ref{exer:303}. }] \label{ans:303}
 \begin{tasks}[column-sep=1em, item-indent=1.3333em](3)
	 \task --1/6
	 \task 0
	 \task --9
	 \task 1
	 \task --12
	 \task $\mp \infty $
	 \task 98/5
\end{tasks}
 \end{enumerate}
\begin{enumerate}


 \needspace{2\baselineskip} 

\phantomsection
 \item[\fontfamily{phv}\selectfont\color{blue}\textbf{\ref{exer:304}. }] \label{ans:304}
 \begin{tasks}[column-sep=1em, item-indent=1.3333em](3)
	 \task $\infty $
	 \task --1/2
	 \task 1/6
	 \task $+\infty $
	 \task $-1/(2\sqrt {3})$
	 \task --1/4
\end{tasks}
 \end{enumerate}
\vspace{0.3cm}

 \needspace{3\baselineskip} 

\hyperlink{page.76}{\textbf{\em Pàgina 76}}
\begin{enumerate}


 \needspace{2\baselineskip} 

\phantomsection
 \item[\fontfamily{phv}\selectfont\color{blue}\textbf{\ref{exer:308}. }] \label{ans:308}
 \begin{tasks}[column-sep=1em, item-indent=1.3333em](3)
	 \task $-\infty $
	 \task 0
	 \task $-3$
	 \task 0
	 \task $-1$
	 \task $-\infty $
	 \task $0$
	 \task $-\infty $
\end{tasks}
 \end{enumerate}
\vspace{0.3cm}

 \needspace{3\baselineskip} 

\hyperlink{page.78}{\textbf{\em Pàgina 78}}
\begin{enumerate}
\phantomsection
\item[\fontfamily{phv}\selectfont\color{blue}\textbf{\ref{exer:311}. }] \label{ans:311} 
\begin{tasks}[counter-format=(tsk[1]),label-width=4ex](2) \task *(2) $\mathop {lim}\limits _{x\to 0^- } f(x)=+\infty $, $\mathop {lim}\limits _{x\to 0^+ } f(x)=-\infty $ \task $-3/2$ \task $-1/4$\startnewitemline \task *(2) $\mathop {lim}\limits _{x\to 1/2^- } f=-\infty $, $\mathop {lim}\limits _{x\to 1/2^+ } f=+\infty $ \task 0\startnewitemline \task *(3) $\mathop {lim}\limits _{x\to 7^- } f(x)=+\infty $, $\mathop {lim}\limits _{x\to 7^+ } f(x)=-\infty $ \task *(2) $\mathop {lim}\limits _{x\to 2^- } f(x)=-\infty $, $\mathop {lim}\limits _{x\to 2^+ } f(x)=+\infty $ \task $+\infty $\startnewitemline \task *(2) $\mathop {lim}\limits _{x\to 1^- } f(x)=-\infty $, $\mathop {lim}\limits _{x\to 1^+ } f(x)=+\infty $ \task 10\startnewitemline \task *(2) $\mathop {lim}\limits _{x\to 5^- } f(x)=-\infty $, $\mathop {lim}\limits _{x\to 5^+ } f(x)=+\infty $ \task 6 \task $+\infty $ \task $-\infty $ \task $+\infty $ \task $0$ \task $0$ \task $0$ \task $0$ \task $2/5$ \task $-7/3$ \task $+\infty $ \task $-\infty $ \task $+\infty $ \task $-1$ \task *(2) $\mathop {lim}\limits _{x\to 0^- } f(x)=+\infty $, $\mathop {lim}\limits _{x\to 0^+ } f(x)=-\infty $ \task $-\infty $ \task $\sqrt {3}$ \task 1/2 \task 0 \task 1/5 \task $+\infty $ \task 3/2 \task 2 \task 0 \task $+\infty $ \task $1/4$\startnewitemline \task *(2) $\mathop {lim}\limits _{x\to 2^- } f(x)=-\infty $, $\mathop {lim}\limits _{x\to 2^+ } f(x)=+\infty $ \task -4 \task 2/3\startnewitemline \task *(2) $\mathop {lim}\limits _{x\to 3^- } f(x)=-\infty $, $\mathop {lim}\limits _{x\to 3^+ } f(x)=+\infty $ \task 1/2 \end{tasks}
 \end{enumerate}
\vspace{0.3cm}

 \needspace{3\baselineskip} 

\hyperlink{page.81}{\textbf{\em Pàgina 81}}
\begin{enumerate}


 \needspace{2\baselineskip} 

\phantomsection
 \item[\fontfamily{phv}\selectfont\color{blue}\textbf{\ref{exer:319}. }] \label{ans:319}
 \begin{tasks}[column-sep=1em, item-indent=1.3333em](1)
	 \task AV. $x=3$ AO. $y=x+1$
	 \task AV. $x=\pm 2$ AH. $y=0$
	 \task AV. $x=- 2$ AH. $y=1$
	 \task AV. $x=\pm 1$ AH. $y=1$
	 \task AV. $x=1$ AH. $y=0$
	 \task AV. $x=1$ BP.
	 \task AV. $x=0$ i $x=1$ BP.
	 \task AV. $x=1$ AH. $y=0$
\end{tasks}
 \end{enumerate}
\vspace{0.3cm}

 \needspace{3\baselineskip} 

\hyperlink{page.84}{\textbf{\em Pàgina 84}}
\begin{enumerate}
\phantomsection
\item[\fontfamily{phv}\selectfont\color{blue}\textbf{\ref{exer:338}. }] \label{ans:338} 
$\limx {1^-} f=+\infty $, $\limx {1^+} f=-\infty $
 \end{enumerate}
\begin{enumerate}
\phantomsection
\item[\fontfamily{phv}\selectfont\color{blue}\textbf{\ref{exer:339}. }] \label{ans:339} 
$\limx {-2^-} f=-\infty $, $\limx {-2^+} f=+\infty $
\phantomsection
\item[\fontfamily{phv}\selectfont\color{blue}\textbf{\ref{exer:340}. }] \label{ans:340} 
$-2/3$
\phantomsection
\item[\fontfamily{phv}\selectfont\color{blue}\textbf{\ref{exer:341}. }] \label{ans:341} 
$1/2$
\phantomsection
\item[\fontfamily{phv}\selectfont\color{blue}\textbf{\ref{exer:342}. }] \label{ans:342} 
a) $5$, b) $+\infty $
\phantomsection
\item[\fontfamily{phv}\selectfont\color{blue}\textbf{\ref{exer:343}. }] \label{ans:343} 
$\infty $
\phantomsection
\item[\fontfamily{phv}\selectfont\color{blue}\textbf{\ref{exer:344}. }] \label{ans:344} 
Té un salt infinit a $x=0$
\phantomsection
\item[\fontfamily{phv}\selectfont\color{blue}\textbf{\ref{exer:345}. }] \label{ans:345} 
Té un salt finit a $x=2$
\phantomsection
\item[\fontfamily{phv}\selectfont\color{blue}\textbf{\ref{exer:346}. }] \label{ans:346} 
$k=2$
\phantomsection
\item[\fontfamily{phv}\selectfont\color{blue}\textbf{\ref{exer:334}. }] \label{ans:334} 
Has d'imposar les condicions de continuïtat en $x=a$; obtindràs l'equació de segon grau $a^2-a-2=0$. Per a $a=-1$ i $a=2$ és continua; discontinuïtat de salt.
 \end{enumerate}

 \vspace{1cm} 
 
 \needspace{5\baselineskip} 
 \heading{Solucions del Tema 7}

\vspace{0.3cm}

 \needspace{3\baselineskip} 

\hyperlink{page.91}{\textbf{\em Pàgina 91}}
\begin{enumerate}


 \needspace{2\baselineskip} 

\phantomsection
 \item[\fontfamily{phv}\selectfont\color{blue}\textbf{\ref{exer:371}. }] \label{ans:371}
 \begin{tasks}[column-sep=1em, item-indent=1.3333em](1)
	 \task $y'=3(x^2+x+1)^2\,(2x+1)$
	 \task* $y'=10\ln ^4 (2x+3) \dfrac {1}{2x+3}$
	 \task $y'=60(3x^4+7)^4 \,x^3$
	 \task* $y'=\dfrac {3x+1}{\sqrt {3x^2+2x+7}}$
	 \task $y'=-2x \,e^{-x^2}$
	 \task* $y'=\cos (\ln x)\,\dfrac {1}{x}$
	 \task* $y'=\dfrac {1}{2\sqrt {x}\,\tg \sqrt {x}}$
	 \task* $y'=\dfrac {2x \left [1+\tg ^2(x^2+1)\right ]}{\tg (x^2+1)}$
\end{tasks}
 \end{enumerate}
\vspace{0.3cm}

 \needspace{3\baselineskip} 

\hyperlink{page.92}{\textbf{\em Pàgina 92}}
\begin{enumerate}


 \needspace{2\baselineskip} 

\phantomsection
 \item[\fontfamily{phv}\selectfont\color{blue}\textbf{\ref{exer:375}. }] \label{ans:375}
 \begin{tasks}[column-sep=1em, item-indent=1.3333em](1)
	 \task $y'=(1-x^2)\sin x+3x\cos x$
	 \task $y'=\ln x +1$
	 \task* $y'=3^x \left ( \tg ^2 x + \ln 3 \tg x + 1 \right )$
	 \task $y'=(x+1)^2 e^x$
	 \task* $y'=10x\,\arctg x + \dfrac {5x^2}{1+x^2}$
	 \task* $y'=\dfrac {\sin x^2}{x+1} + 2x\ln x\cos x^2$
\end{tasks}
 \end{enumerate}
\begin{enumerate}


 \needspace{2\baselineskip} 

\phantomsection
 \item[\fontfamily{phv}\selectfont\color{blue}\textbf{\ref{exer:376}. }] \label{ans:376}
 \begin{tasks}[column-sep=1em, item-indent=1.3333em](1)
	 \task $y'=\frac {1 - \ln x }{x^2}$
	 \task $y'=\frac {-1}{\sin ^2 x}$
	 \task $y'=\frac {-2}{(x-1)^2}$
	 \task* $y'=\frac {4x-3\sqrt {x^3}}{2x^4}$
	 \task* $y'=\frac {-2\cos x}{(1+\sin x)^2}$
	 \task $y'=\frac {x^2+1}{(x^2-1)^2}$
	 \task* $y'=\frac {-4 \; x^{2} + 20 \; x - 22}{(x-3)^2 (x-4)^2}$
	 \task* $y'=\frac {-3 \; x^{2} - 6 \; x - 3}{(x^3+3x^2+3x-1)^2}$
\end{tasks}
 \end{enumerate}
\vspace{0.3cm}

 \needspace{3\baselineskip} 

\hyperlink{page.102}{\textbf{\em Pàgina 102}}
\begin{enumerate}
\phantomsection
\item[\fontfamily{phv}\selectfont\color{blue}\textbf{\ref{exer:406}. }] \label{ans:406} 
Minimitzeu la funció $f(x)=5x^2+6(44-x)^2$. Trobareu $x=22$.
 \end{enumerate}
\begin{enumerate}
\phantomsection
\item[\fontfamily{phv}\selectfont\color{blue}\textbf{\ref{exer:407}. }] \label{ans:407} 
$V(x)=(25-2x)\cdot (20-2x)\cdot x$. Màxim és $x=3.68$ cm, $V=820.53$ cm$^{3}$.
 \end{enumerate}
\vspace{0.3cm}

 \needspace{3\baselineskip} 

\hyperlink{page.103}{\textbf{\em Pàgina 103}}
\begin{enumerate}
\phantomsection
\item[\fontfamily{phv}\selectfont\color{blue}\textbf{\ref{exer:415}. }] \label{ans:415} 
$f'(1)=1$
 \end{enumerate}
\begin{enumerate}
\phantomsection
\item[\fontfamily{phv}\selectfont\color{blue}\textbf{\ref{exer:416}. }] \label{ans:416} 
$m=f'(2)=-16/121$
\phantomsection
\item[\fontfamily{phv}\selectfont\color{blue}\textbf{\ref{exer:417}. }] \label{ans:417} 
$y'=2x \cdot 2^{x^2+3} \, \ln 2$ 
\phantomsection
\item[\fontfamily{phv}\selectfont\color{blue}\textbf{\ref{exer:418}. }] \label{ans:418} 
$y'=-6x^2\cos x^3 \, \sin x^3 $
\phantomsection
\item[\fontfamily{phv}\selectfont\color{blue}\textbf{\ref{exer:419}. }] \label{ans:419} 
$y = 2x + 6$
\phantomsection
\item[\fontfamily{phv}\selectfont\color{blue}\textbf{\ref{exer:420}. }] \label{ans:420} 
$y=0$
\phantomsection
\item[\fontfamily{phv}\selectfont\color{blue}\textbf{\ref{exer:421}. }] \label{ans:421} 
\textit {x}$<$ 0, creixent; 0 $<$ \textit {x}$<$ 4, decreixent; \textit {x}$>$ 4, creixent
\phantomsection
\item[\fontfamily{phv}\selectfont\color{blue}\textbf{\ref{exer:422}. }] \label{ans:422} 
$(0, 0)$ mínim i $(1, 1)$ màxim relatius
\phantomsection
\item[\fontfamily{phv}\selectfont\color{blue}\textbf{\ref{exer:423}. }] \label{ans:423} 
Punt d'inflexió $(3, -45)$; recta tangent $y=-24x+27$
 \end{enumerate}

 \vspace{1cm} 

 \needspace{5\baselineskip} 
 \heading{Solucions del Bloc II}

\vspace{0.3cm}

 \needspace{3\baselineskip} 

\hyperlink{page.104}{\textbf{\em Pàgina 104}}
\begin{enumerate}
\phantomsection
\item[\fontfamily{phv}\selectfont\color{blue}\textbf{\ref{exer:424}. }] \label{ans:424} 
a) $\mathbb {R}-\{0, -5\}$, b) $(-\infty , 5/2]$
 \end{enumerate}
\begin{enumerate}
\phantomsection
\item[\fontfamily{phv}\selectfont\color{blue}\textbf{\ref{exer:425}. }] \label{ans:425} 
 a) \includegraphics *[width=0.22\textwidth ]{img-07-bloc2/bloc2-2a.png} \par b) \includegraphics *[width=0.22\textwidth ]{img-07-bloc2/bloc2-2b.png}
\phantomsection
\item[\fontfamily{phv}\selectfont\color{blue}\textbf{\ref{exer:426}. }] \label{ans:426} 
$g(h(x))=\sin \sqrt {x}$, $f(g(x))=e^{\sin x}$, $h(f(x))=\sqrt {e^x}$. $h(f(x))^{-1}=\ln x^2$.
\phantomsection
\item[\fontfamily{phv}\selectfont\color{blue}\textbf{\ref{exer:427}. }] \label{ans:427} 
a) $4$, b) $\limx {-5^-} f(x)=+\infty $ i $\limx {-5^+} f(x)=-\infty $ , c) $-\infty $
\phantomsection
\item[\fontfamily{phv}\selectfont\color{blue}\textbf{\ref{exer:428}. }] \label{ans:428} 
a) $b=1$, b) $f$ és discontínua a $x=2$ per punt desplaçat.
\phantomsection
\item[\fontfamily{phv}\selectfont\color{blue}\textbf{\ref{exer:429}. }] \label{ans:429} 
$f'(3)=\limx [h]{0} \frac {f(3+h)-f(3)}{h}= \limx [h]{0} \frac {3(3+h)^2-10(3+h)-(-3)}{h} =8$.
\phantomsection
\item[\fontfamily{phv}\selectfont\color{blue}\textbf{\ref{exer:430}. }] \label{ans:430} 
a) $y(0)=100$ bacteris, \par $y(30)=100\cdot e^{0.05\cdot 30}\approx 448$; \par b) $5000 = 100\cdot e^{0.05\cdot t}$ $\rightarrow $ $0.05\cdot t = \ln (5000/100)$ $\rightarrow $ $t = \ln (50)/0.05 \approx 78,2 $ \par c) Gràfica \par \includegraphics *[width=0.4\textwidth ]{img-07-bloc2/bloc2-6.png} 
 \end{enumerate}
\vspace{0.3cm}

 \needspace{3\baselineskip} 

\hyperlink{page.105}{\textbf{\em Pàgina 105}}
\begin{enumerate}
\phantomsection
\item[\fontfamily{phv}\selectfont\color{blue}\textbf{\ref{exer:431}. }] \label{ans:431} 
$y=-\frac {2}{3}+\frac {5}{9}(x-1)$ o $y=\frac {5x}{9}-\frac {11}{9}$.
 \end{enumerate}
\begin{enumerate}
\phantomsection
\item[\fontfamily{phv}\selectfont\color{blue}\textbf{\ref{exer:432}. }] \label{ans:432} 
\begin{tasks} \task $y'=-\frac {1}{2\sqrt {\cos (5x^4+2x^3)}}\cdot \sin (5x^4+2x^3) \cdot (20x^3+6x^2)$ \task $y'=\frac {1-2\ln x}{x^3}$ \task $y'=\frac {1}{\sqrt {1-(3x^5-6x^2)^2}}\cdot (15x^4-12x)$ \task $y'=(1-2x^2)\cdot e^{-x^2}$ \task $y'=-85\left ( \frac {2x+5}{3x-1} \right )^4 \cdot \frac {1}{(3x-1)^2}$ \task $y'=\frac {1}{2\sqrt {x+\sqrt {x}}}\cdot (1+\frac {1}{2\sqrt {x}})$ \end{tasks}
\phantomsection
\item[\fontfamily{phv}\selectfont\color{blue}\textbf{\ref{exer:433}. }] \label{ans:433} 
a) Creixent: $(-\infty , -2)\cup (2,+\infty )$; Decreixent $(-2,2)$; Màxims: $(x=-2,\, y=16)$; Mínims: $(x=2, y=-16)$. b)Sempre és creixent. No té extrems.
\phantomsection
\item[\fontfamily{phv}\selectfont\color{blue}\textbf{\ref{exer:434}. }] \label{ans:434} 
Punts de tall amb l'eix OX: $x=-2$, $x=0$, $x=3$, i amb l'eix OY $(0,0)$. Té un màxim a (0,0) i té dos mínims a $x=2.15$, $y=-16.3$ i a $x=-1.4$, $y=-5.2$ La gràfica de la funció és: \par \includegraphics [width=0.3\textwidth ]{img-07-bloc2/sol-bloc2-polinomi1.png} 
\phantomsection
\item[\fontfamily{phv}\selectfont\color{blue}\textbf{\ref{exer:435}. }] \label{ans:435} 
\begin{tasks} \task $y=1$ asímptota horitzontal; $x=1$ i $x=3$ asímptotes verticals. La posició relativa és: $\limx {+\infty } f(x)=1$ per damunt; $\limx {-\infty } f(x)=1$ per davall. $\limx {1^-} f(x)=+\infty $, $\limx {1^+} f(x)=-\infty $, $\limx {3^-} f(x)=-\infty $, $\limx {3^+} f(x)=+\infty $. \task La funció té un mínim relatiu a $x=0$, $y=0$ i un màxim relatiu a $x=3/2$, $y=-3$ \task Gràfica: \par \includegraphics *[width=0.35\textwidth ]{img-07-bloc2/bloc2-10.png} \end{tasks}
\phantomsection
\item[\fontfamily{phv}\selectfont\color{blue}\textbf{\ref{exer:436}. }] \label{ans:436} 
L'única funció que presenta asímptota obliqua és c). L'asímptota obliqua és $y=2x+2$ i també té una asímptota vertical a $x=1$. La gràfica és la següent: \par \includegraphics *[width=0.4\textwidth ]{img-07-bloc2/bloc2-11.png} 
\phantomsection
\item[\fontfamily{phv}\selectfont\color{blue}\textbf{\ref{exer:437}. }] \label{ans:437} 
 $k=1/2$. La primera derivada $y'=(2x^2+2x+1)/(x+1/2)^2$ mai és zero. Sempre creix i per tant no té extrems.
\phantomsection
\item[\fontfamily{phv}\selectfont\color{blue}\textbf{\ref{exer:438}. }] \label{ans:438} 
$a=-6$ i $b=17$.
 \end{enumerate}

 \vspace{1cm} 
 
 \needspace{5\baselineskip} 
 \heading{Solucions del Tema 8}

\vspace{0.3cm}

 \needspace{3\baselineskip} 

\hyperlink{page.114}{\textbf{\em Pàgina 114}}
\begin{enumerate}
\phantomsection
\item[\fontfamily{phv}\selectfont\color{blue}\textbf{\ref{exer:491}. }] \label{ans:491} 
$(3/5, -4/5)$ i $(4/5, 3/5)$.
 \end{enumerate}
\vspace{0.3cm}

 \needspace{3\baselineskip} 

\hyperlink{page.115}{\textbf{\em Pàgina 115}}
\begin{enumerate}


 \needspace{2\baselineskip} 

\phantomsection
 \item[\fontfamily{phv}\selectfont\color{blue}\textbf{\ref{exer:494}. }] \label{ans:494}
 \begin{tasks}[column-sep=1em, item-indent=1.3333em](2)
	 \task $105.07^\circ $
	 \task $180^\circ $
\end{tasks}
 \end{enumerate}
\begin{enumerate}
\phantomsection
\item[\fontfamily{phv}\selectfont\color{blue}\textbf{\ref{exer:495}. }] \label{ans:495} 
$x=-1$, $\alpha =57.53^\circ $
 \end{enumerate}
\vspace{0.3cm}

 \needspace{3\baselineskip} 

\hyperlink{page.117}{\textbf{\em Pàgina 117}}
\begin{enumerate}
\phantomsection
\item[\fontfamily{phv}\selectfont\color{blue}\textbf{\ref{exer:521}. }] \label{ans:521} 
a) Punt final $(-2, 3)$. b) Vector suma $\vvec +\vec w=(-2, 3)$.
 \end{enumerate}
\begin{enumerate}


 \needspace{2\baselineskip} 

\phantomsection
 \item[\fontfamily{phv}\selectfont\color{blue}\textbf{\ref{exer:522}. }] \label{ans:522}
 \begin{tasks}[column-sep=1em, item-indent=1.3333em](2)
	 \task $-3$
	 \task $-11$
\end{tasks}
\phantomsection
\item[\fontfamily{phv}\selectfont\color{blue}\textbf{\ref{exer:523}. }] \label{ans:523} 
 a) $2\sqrt {3}$. b) Els dos tenen mòdul 2. c) angle $30^\circ $
\phantomsection
\item[\fontfamily{phv}\selectfont\color{blue}\textbf{\ref{exer:524}. }] \label{ans:524} 
a) $k=-2$. b) $k=\pm 4$. c) $k=-\sqrt {3}$
\phantomsection
\item[\fontfamily{phv}\selectfont\color{blue}\textbf{\ref{exer:525}. }] \label{ans:525} 
$(3/5, 4/5)$ o $(-3/5, -4/5)$
 \end{enumerate}

 \vspace{1cm} 
 
 \needspace{5\baselineskip} 
 \heading{Solucions del Tema 9}

\vspace{0.3cm}

 \needspace{3\baselineskip} 

\hyperlink{page.125}{\textbf{\em Pàgina 125}}
\begin{enumerate}
\phantomsection
\item[\fontfamily{phv}\selectfont\color{blue}\textbf{\ref{exer:550}. }] \label{ans:550} 
El feix de rectes és $r:$ $y-2=m(x-1)$. Passa-la a forma general i aplica que $d(r,O)=1$.
 \end{enumerate}
\vspace{0.3cm}

 \needspace{3\baselineskip} 

\hyperlink{page.127}{\textbf{\em Pàgina 127}}
\begin{enumerate}
\phantomsection
\item[\fontfamily{phv}\selectfont\color{blue}\textbf{\ref{exer:567}. }] \label{ans:567} 
Escriu el feix de rectes com $y=1-m(x-3)$, troba els punts de tall amb els eixos i comprova que l'àrea és $A=\frac {1}{2}(1+3m)(3+\frac {1}{m})=6$. Resol l'equació i troba $m=1/3$.
 \end{enumerate}
\vspace{0.3cm}

 \needspace{3\baselineskip} 

\hyperlink{page.128}{\textbf{\em Pàgina 128}}
\begin{enumerate}
\phantomsection
\item[\fontfamily{phv}\selectfont\color{blue}\textbf{\ref{exer:570}. }] \label{ans:570} 
Troba el peu de la perpendicular de $r$ pel punt $A$: $B=(0,3)$, $C=(1,4)$ i $D=(2,3)$ 
 \end{enumerate}
\vspace{0.3cm}

 \needspace{3\baselineskip} 

\hyperlink{page.129}{\textbf{\em Pàgina 129}}
\begin{enumerate}
\phantomsection
\item[\fontfamily{phv}\selectfont\color{blue}\textbf{\ref{exer:581}. }] \label{ans:581} 
 a) $k=-7$, b) $k=-7$
 \end{enumerate}
\begin{enumerate}
\phantomsection
\item[\fontfamily{phv}\selectfont\color{blue}\textbf{\ref{exer:582}. }] \label{ans:582} 
Contínua $\frac {x-3}{5}=\frac {y-2}{1}$, general $x-5y+7=0$.
\phantomsection
\item[\fontfamily{phv}\selectfont\color{blue}\textbf{\ref{exer:583}. }] \label{ans:583} 
a) $(x,y)=(2,-3)+\lambda (2, 5)$, b) $2x+3y-6=0$.
\phantomsection
\item[\fontfamily{phv}\selectfont\color{blue}\textbf{\ref{exer:584}. }] \label{ans:584} 
Si $k=-9/5$ són paral·leles, en altre cas són secants.
\phantomsection
\item[\fontfamily{phv}\selectfont\color{blue}\textbf{\ref{exer:585}. }] \label{ans:585} 
$k=\pm \sqrt {3}$.
\phantomsection
\item[\fontfamily{phv}\selectfont\color{blue}\textbf{\ref{exer:586}. }] \label{ans:586} 
$x=-3$, $y=-5$ i $x=17/5$ i $y=39/5$
 \end{enumerate}

 \vspace{1cm} 
 
 \needspace{5\baselineskip} 
 \heading{Solucions del Tema 10}

\vspace{0.3cm}

 \needspace{3\baselineskip} 

\hyperlink{page.132}{\textbf{\em Pàgina 132}}
\begin{enumerate}
\phantomsection
\item[\fontfamily{phv}\selectfont\color{blue}\textbf{\ref{exer:587}. }] \label{ans:587} 
Radi $\sqrt {8}$, $(x+1)^2+(y-3)^2=8$
 \end{enumerate}
\begin{enumerate}
\phantomsection
\item[\fontfamily{phv}\selectfont\color{blue}\textbf{\ref{exer:588}. }] \label{ans:588} 
Centre $O(1,0)$, radi $R=1$
 \end{enumerate}
\vspace{0.3cm}

 \needspace{3\baselineskip} 

\hyperlink{page.133}{\textbf{\em Pàgina 133}}
\begin{enumerate}
\phantomsection
\item[\fontfamily{phv}\selectfont\color{blue}\textbf{\ref{exer:589}. }] \label{ans:589} 
Centre $O(-1,1)$, semi-eixos $a=3$, $b=2$, focus $F'(-1-\sqrt {5}, 1)$ i $F'(-1+\sqrt {5}, 1)$. Excentricitat $e=0.745$
 \end{enumerate}
\begin{enumerate}
\phantomsection
\item[\fontfamily{phv}\selectfont\color{blue}\textbf{\ref{exer:590}. }] \label{ans:590} 
$\frac {x^2}{9}+\frac {y^2}{5}=1$ $e=2/3$
\phantomsection
\item[\fontfamily{phv}\selectfont\color{blue}\textbf{\ref{exer:591}. }] \label{ans:591} 
$\frac {x^2}{100}+\frac {y^2}{96}=1$
 \end{enumerate}
\vspace{0.3cm}

 \needspace{3\baselineskip} 

\hyperlink{page.134}{\textbf{\em Pàgina 134}}
\begin{enumerate}
\phantomsection
\item[\fontfamily{phv}\selectfont\color{blue}\textbf{\ref{exer:592}. }] \label{ans:592} 
$O(1,0)$, $a=4$, $b=3$, $F'(-4,0)$ i $F(6,0)$, asímptotes $y=\pm 3(x-1)/4$, $e=5/4$.
 \end{enumerate}
\begin{enumerate}
\phantomsection
\item[\fontfamily{phv}\selectfont\color{blue}\textbf{\ref{exer:593}. }] \label{ans:593} 
$a=b=1/\sqrt {2}$, $2{x^2}-2(y-2)^2=1$, $e=\sqrt {2}$.
 \end{enumerate}
\vspace{0.3cm}

 \needspace{3\baselineskip} 

\hyperlink{page.135}{\textbf{\em Pàgina 135}}
\begin{enumerate}
\phantomsection
\item[\fontfamily{phv}\selectfont\color{blue}\textbf{\ref{exer:594}. }] \label{ans:594} 
$x=\frac {1}{12}y^2$
 \end{enumerate}
\begin{enumerate}
\phantomsection
\item[\fontfamily{phv}\selectfont\color{blue}\textbf{\ref{exer:595}. }] \label{ans:595} 
$V(3,0)$, $F(0, 1)$, $d: y=-1$
\phantomsection
\item[\fontfamily{phv}\selectfont\color{blue}\textbf{\ref{exer:596}. }] \label{ans:596} 
És una hipèrbola. $d=c-a=0.75 a$
 \end{enumerate}
\vspace{0.3cm}

 \needspace{3\baselineskip} 

\hyperlink{page.137}{\textbf{\em Pàgina 137}}
\begin{enumerate}
\phantomsection
\item[\fontfamily{phv}\selectfont\color{blue}\textbf{\ref{exer:598}. }] \label{ans:598} 
$x^2+y^2-2x-2y-23=0$ o $(x-1)^2+(y-1)^2=25$. Té centre $O(1,1)$ i radi $R=5$.
 \end{enumerate}
\begin{enumerate}
\phantomsection
\item[\fontfamily{phv}\selectfont\color{blue}\textbf{\ref{exer:599}. }] \label{ans:599} 
$\frac {(x+1)^2}{25}+\frac {(y-3)^2}{9}=1$. La semi-distància focal $c=4$, els focus són $F'(-5,3)$ i $F(3,3)$, i l'excentricitat $e=0.8$.
\phantomsection
\item[\fontfamily{phv}\selectfont\color{blue}\textbf{\ref{exer:600}. }] \label{ans:600} 
Semi-eixos: $a=2$, $b=\sqrt {2}$, les asímptotes $y=\pm \frac {\sqrt {2}}{2}x$, semi-distància focal: $c=\sqrt {6}$ i l'excentricitat $e=1.225$.
\phantomsection
\item[\fontfamily{phv}\selectfont\color{blue}\textbf{\ref{exer:601}. }] \label{ans:601} 
La distància focus-directriu és $p=1/6$, l'equació $y-2=3 (x-1)^2$, la directriu és la recta $y=23/12$ i la posició del focus $F(1, 25/12)$.
\phantomsection
\item[\fontfamily{phv}\selectfont\color{blue}\textbf{\ref{exer:602}. }] \label{ans:602} 
a) El·lipse de centre $(1,0)$ i semi-eixos $a=2$, $b=1$. b) Circumferència de centre $(1,2)$ i radi $2$. c) Paràbola vertical de vèrtex $(0,-2)$ i distància Focus-directriu $p=3/2$.
 \end{enumerate}

 \vspace{1cm} 

 \needspace{5\baselineskip} 
 \heading{Solucions del Bloc III}

\vspace{0.3cm}

 \needspace{3\baselineskip} 

\hyperlink{page.138}{\textbf{\em Pàgina 138}}
\begin{enumerate}
\phantomsection
\item[\fontfamily{phv}\selectfont\color{blue}\textbf{\ref{exer:603}. }] \label{ans:603} 
a) $|\vec u|=\sqrt {2}$ b) $-2\vec u + 3\vvec =(-5,-4)$ c) $2\vec u \cdot (\vec u + \vvec )=6$
 \end{enumerate}
\begin{enumerate}
\phantomsection
\item[\fontfamily{phv}\selectfont\color{blue}\textbf{\ref{exer:604}. }] \label{ans:604} 
 $a=-2$
\phantomsection
\item[\fontfamily{phv}\selectfont\color{blue}\textbf{\ref{exer:605}. }] \label{ans:605} 
a) $m=-1$, $n=3$ b) $116.57^\circ $
\phantomsection
\item[\fontfamily{phv}\selectfont\color{blue}\textbf{\ref{exer:606}. }] \label{ans:606} 
$(\frac {1}{2}, \frac {\sqrt {3}}{2})$
\phantomsection
\item[\fontfamily{phv}\selectfont\color{blue}\textbf{\ref{exer:607}. }] \label{ans:607} 
$y=-8$
\phantomsection
\item[\fontfamily{phv}\selectfont\color{blue}\textbf{\ref{exer:608}. }] \label{ans:608} 
Paramètriques $\left \{\begin {array}{l} x=0+1t \\ y=3+2t \end {array}\right .$. General $2x-y+3=0$.
\phantomsection
\item[\fontfamily{phv}\selectfont\color{blue}\textbf{\ref{exer:609}. }] \label{ans:609} 
a) $k=-2$. b) $k=-4$
\phantomsection
\item[\fontfamily{phv}\selectfont\color{blue}\textbf{\ref{exer:610}. }] \label{ans:610} 
$A'(2,2)$
 \end{enumerate}
\vspace{0.3cm}

 \needspace{3\baselineskip} 

\hyperlink{page.139}{\textbf{\em Pàgina 139}}
\begin{enumerate}
\phantomsection
\item[\fontfamily{phv}\selectfont\color{blue}\textbf{\ref{exer:611}. }] \label{ans:611} 
El punt d'intersecció és $I(5,11)$ i el pendent de la recta $m'=1/3$. La recta és $y-11=\frac {1}{3}(x-5)$.
 \end{enumerate}
\begin{enumerate}
\phantomsection
\item[\fontfamily{phv}\selectfont\color{blue}\textbf{\ref{exer:612}. }] \label{ans:612} 
$P(\pm 5, 0)$
\phantomsection
\item[\fontfamily{phv}\selectfont\color{blue}\textbf{\ref{exer:613}. }] \label{ans:613} 
$Area=5$
\phantomsection
\item[\fontfamily{phv}\selectfont\color{blue}\textbf{\ref{exer:614}. }] \label{ans:614} 
Mediatriu AC: $x-2y+1=0$, Mediatriu $AB$: $14x-4y+21=0$, Circumcentre $O(-19/12, -7/24)$
\phantomsection
\item[\fontfamily{phv}\selectfont\color{blue}\textbf{\ref{exer:615}. }] \label{ans:615} 
Centre $O(1,-3)$ i radi $R=2$.
\phantomsection
\item[\fontfamily{phv}\selectfont\color{blue}\textbf{\ref{exer:616}. }] \label{ans:616} 
$\frac {x^2}{100}+\frac {y^2}{36}=1$
\phantomsection
\item[\fontfamily{phv}\selectfont\color{blue}\textbf{\ref{exer:617}. }] \label{ans:617} 
 a) Paràbola vertical amb vèrtex $V(0,0)$, $p=3$, focus $F(0,3/2)$ i directriu la recta $y=-3/2$. \par b) Hipèrbola de centre $O(1,-1)$, semieixos $a=4$, $b=3$, semi-distància focal $c=5$. Excentricitat $e=1.25$. Focus a $F'(-4,-1)$, $F(6,-1)$. Asímptotes $y+1=\pm \frac {3}{4}(x-1)$. \par \begin {center} \includegraphics [height=3cm]{img-10-bloc3/bloc3-sol-14a} \includegraphics [height=3cm]{img-10-bloc3/bloc3-sol-14b} \end {center} 
\phantomsection
\item[\fontfamily{phv}\selectfont\color{blue}\textbf{\ref{exer:618}. }] \label{ans:618} 
$a=39.3$ ua, $c=9.8$ ua, $b=38.06$ ua. $\dfrac {x^2}{39.3^2}+\dfrac {y^2}{38.06^2}=1$
 \end{enumerate}

 \vspace{1cm} 
 
 \needspace{5\baselineskip} 
 \heading{Solucions del Tema 11}

\vspace{0.3cm}

 \needspace{3\baselineskip} 

\hyperlink{page.143}{\textbf{\em Pàgina 143}}
\begin{enumerate}
\phantomsection
\item[\fontfamily{phv}\selectfont\color{blue}\textbf{\ref{exer:620}. }] \label{ans:620} 
\begin {tabular}{c|c|c|c|c|c|c|c}\hline $x$ & 0 & 1 & 2 & 3 & 4 & 5 & 6 \\\hline $f$ & 2 &4 & 21 & 15 & 6 & 1 &1 \\ \end {tabular} \par c) $\bar x =2.52$ i $\sigma =0.496$ fills. 
 \end{enumerate}
\vspace{0.3cm}

 \needspace{3\baselineskip} 

\hyperlink{page.144}{\textbf{\em Pàgina 144}}
\begin{enumerate}
\phantomsection
\item[\fontfamily{phv}\selectfont\color{blue}\textbf{\ref{exer:621}. }] \label{ans:621} 
\begin {tabular}{c|c|c|c|c|c}\hline $x$ & 2.5-3 & 3-3.5 & 3.5-4 & 4-4.5 & 4.5-5\\\hline $f$ & 6 & 10 & 11 & 8 & 5 \\ \end {tabular} \par c) $\bar x =3.7$ i $\sigma =0.62$ kg. 
 \end{enumerate}
\vspace{0.3cm}

 \needspace{3\baselineskip} 

\hyperlink{page.148}{\textbf{\em Pàgina 148}}
\begin{enumerate}
\phantomsection
\item[\fontfamily{phv}\selectfont\color{blue}\textbf{\ref{exer:627}. }] \label{ans:627} 
$y=1.194+4.78(x-0.232)$, el pendent és la constant elàstica $k=4.78$ N/m. L'allargament per a $y=2$ N és $x=40$ cm. És bastant fiable ja que $r=0,998$.
 \end{enumerate}
\vspace{0.3cm}

 \needspace{3\baselineskip} 

\hyperlink{page.149}{\textbf{\em Pàgina 149}}
\begin{enumerate}
\phantomsection
\item[\fontfamily{phv}\selectfont\color{blue}\textbf{\ref{exer:629}. }] \label{ans:629} 
a) $r=-0.997$ correlació negativa forta; \quad b) Recta de regressió $y=-0.2632 x+10.37$, altura $y=8.84$ m. \quad c) $x=66$ hores
 \end{enumerate}
\vspace{0.3cm}

 \needspace{3\baselineskip} 

\hyperlink{page.152}{\textbf{\em Pàgina 152}}
\begin{enumerate}
\phantomsection
\item[\fontfamily{phv}\selectfont\color{blue}\textbf{\ref{exer:643}. }] \label{ans:643} 
a) Hi ha una correlació lineal positiva forta $\sigma _{xy}=1.71$ i $r=0.91$\par b) $y=0.87x+2.25$. Es cometen 4.86 errors.
 \end{enumerate}
\begin{enumerate}
\phantomsection
\item[\fontfamily{phv}\selectfont\color{blue}\textbf{\ref{exer:644}. }] \label{ans:644} 
Hi ha una correlació positiva molt feble. $\sigma _{xy}=6.8$, $r=0.60$, $y=4x-95.3$
\phantomsection
\item[\fontfamily{phv}\selectfont\color{blue}\textbf{\ref{exer:645}. }] \label{ans:645} 
a) $y=0.1x+6.65$\quad b) No seria gens fiable fer prediccions en aquest cas ja que $r=0.16$ és molt inferior a 1.
 \end{enumerate}